\subsection{Recursive Types and Functions [Simone]}
\label{sec:Insieme.Frontend.Recursion}

One of the most difficult task of the frontend is the conversion of recursive
entities like function calls and types. As already explained in the overview
section, in C those entities are addressed by their name and therefore once a
symbol has been defined, it is possible to refer to it. In INSPIRE this is not
allowed as entities are referred by their structure, and because at the moment
of creation, all the elements forming an entity must be available, creation of
recursive types is impossible. 

To solve this issues a special type and lambda construct has been introduced in
INSPIRE to allow recursive structure to be represented. The main idea behind it
is that the recursive part of a function, or a type is identified by a name and
when recursion is encountered, the name is used to refer to it. The way we deal
with recursive data types and function is the same. The code implementing this
conversion is in the \decl{convertFunctionDecl()} and \decl{visitTagType()}
functions respectively defined in \file{frontend/expr\_convert.cpp} and
\file{frontend/type\_convert.cpp}. The process is divided into the following
main steps:

\begin{description}

\item [Add Type/Function to Graph:] For every struct/class type or function
which gets converted, we add the entity to respectively a type or call graph. If
the type or function is already in the graph then returned the cached IR node
previously generated.

\item[Compute strong connected components:] If there was a change in the
type/call graph, then we run the algorithm to compute strong connected components
on the new type/call graph. In order to minimize development type, we used the
Boost.Graph library which provides a highly optimized library for representing
graphs and additionally implements several algorithms are already implemented on
it. In order to simplify the use of the Boost library we developed a small
wrapper class \type{frontend::utils::De\-pen\-den\-cyGrap} defined in
\file{frontend/utils/dep\_graph.h}. The class is generic so it can be used for
solving connected components for both functions and types. 

\item[Resolve sub-components:] The strongly connected components algorithm
returns a set of components, or cycles, detected in the given graph. We then
identify the ID of the component to which the current type/function belongs and
proceed with the conversion of all the component with a smaller ID. 

\item[Resolve current function:] Once we are sure all sub-cycles are converted
into INSPIRE lambdas/types, we proceed with the resolution of the current
function/type. Before invoking the visitor on the body of the function however
we have to change the context of the \type{frontend::ConvertionManager} to make
it aware that we are converting a recursive function, (i.e. {\tt isRecSubFunc}
for functions and {\tt isRecSubType} for types). This is important because
if in the function body we encounter a recursive call, then we should avoid to
recur again. This is done by using a boolean variable which indicates the
converter that we are resolving something which has been detected to be
recursive. 
\begin{itemize}

	\item Before recurring, and solving the function body or the stucture type
		field types, we define a set of variables (one for each entity in the
		current component) which represent a placeholder for that particular
		entity. For functions the map which associates variable names to the
		corresponding function declaration is called {\tt varDeclMap}. While for
		types the map is called {\tt recVarMap}.

	\item We then recur over the body. When a new call expression, or a struct
		type is encountred we determine whether we are already resolving a
		recursive function/type, if yes we determine whether the current
		entity belongs to the connected component. In order to optimize the
		execution time, instead of re-computing the connected components, we
		lookup the map containing the variables being associated to the elements
		of the current component. Otherwise we continue with the normal
		execution. 

	\item When the body has been converted we reset the context variable to the
		normal execution and we store in a function/type cache the converted
		IR expression to be associated to the corresponding Clang type/function.

	\item Because the definition of a recursive function or type include
		definitions also for other elements in the component, at this point we
		store in the cache the corresponding INSPIRE node who represent the
		other elements in the connected component. 

\end{itemize}

\end{description}

The algorithm might be complex but it has been tested with multiple scenario and
it has been proved to be sound. The concern of the future developer of insieme
should be extendibility to new entities. For example the current implementation
works for regular call expressions. However if CXX support needs to be ported
into insieme, the call graph needs to be extended to take into account also
constructurs, destructors, method invocations and operator overalods. This can
be done using a type trait trick. The \type{frontend::utils::DependencyGraph}
provides a \decl{Handle} method which should be extendend to handle specific
Clang nodes. This methos tells the dependency graph builder how explore a new
element added to the graph. For example if a new call expression is added, we
should add to the dependency graph eventual methods invoked within the body of
the function. 

