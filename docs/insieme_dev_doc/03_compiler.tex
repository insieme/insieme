\chapter{The Insieme Compiler} \label{cap:compiler}
\index{Compiler}

\section{Overview}

To be noted:
\begin{itemize}
  \item The compiler is a library, not a framework or a single executable with a
  tone of flags
  \item Overview on the various modules
\end{itemize}

\subsection{Architecture}
\subsection{Coding Standards}
\subsection{Source Code Organization}
\subsection{Building a Simple Optimizing Compiler}
\label{cap:compiler:sec:overview:sub:building} TODO: add an example loading
code, changing something stupid and generating output code.

\section{The Utilities}

\subsection{Command Line Arguments [Simone]}

\label{Command.Line.Args}

A compiler usually comes with a number of input options which can be used to
drive the behaviour of the compiler. For example, by enabling the user to
trigger different analysis or optimizations during the compilation process.
In Insieme input arguments are managed by the {\tt Boost.Program\_options}
library \cite{boost_program_options}. However, in order to make the operation of
adding new input arguments/flags to the compiler easy, the library has been
wrapped into a framework which allows the specification of input flags via a
simple XMacro files. 

The file \file{utils/cmd\_line\_utils.h} contains the definition of the
\type{utils::CommandLineOptions} class. This class is populated by the C preprocessor
using the content of the XMacro file \file{utils/options.def}. Each line of the
file contains the definition of an input argument of the Insieme compiler. The
file is divided into 3 sections containing respectively: 

\begin{description}
\item [Flags]: A flag can appear or not appear in the command line. When it
appears its value is set to the boolean value true, otherwise its value is
false. An example is the {\tt --help or -h} or the {\tt --check-sema or -S}
flags. 

\item [Integer Options]: This is an optional argument which can assume an
integer value. When the option does not appear in the input argument it is set
to a default value. An example is the option used to set the verbosity level of
the compiler, i.e. {\tt -v 1} or {\tt --verbose=1}. 

\item [Options]: Very similar to the Integer counter-part but the value of an
option can be either a string or a vector of strings.
\end{description}

An example of the structure of the \file{utils/options.def} XMacro file is the
following:

\begin{insCode} 
/* FLAG(opt_name, var_name, def_value, var_help) */
FLAG("help,h",    Help,     false,     "produce help message")
...
/* INT_OPTION(opt_name, var_name, def_value, var_help) */
INT_OPTION("verbose,v", Verbosity, 0, "Set verbosity level")
...
/* OPTION(opt_name, var_name, var_type, var_help) */ 
OPTION("out-file,o", Output,  std::string, "output file")

OPTION("include-path,I", IncludePaths, std::vector<std::string>, 	
	   "Add directory to include search path")
...
\end{insCode}

An input flag for the compiler is defined by the {\tt FLAG()} macro. This macro
accepts 4 arguments, all of them are mandatory and none is optional. 
\begin{itemize}
\item The first argument is the name of the flag. Two versions are provided:
full and abbreviated. For example the definition {\tt "help,h"} defines two
input arguments in the form {\tt -h} or {\tt --help} both connected to the same
input flag.  

\item The second argument is the variable name which will be
utilized to store the value of the flag. The name of this variable MUST be
unique. 

\item The third argument is the default value of the which the flag should
assume when not specified in the command line. Usually you want to set this
value to false, however some flags can be forced to be set to true by default. 

\item The last argument is a string literal containing the message which explain
the semantics of the defined flag. This message is shown whenever the {\tt
--help} option is invoked on the Insieme compiler executable. 
\end{itemize}

Definition of an integer option is similar to the one for flags. The only
difference is that the default value can be any integer value. For generic
options we do not provide, for now, the ability of specifying the default value
but instead the third argument is utilized to specify the type associated to
that option.  Valid types for options can be \type{std::string},
\type{std::vector<std::string>}, \type{int}, \type{std::vector<int>}. List
options capture the use case where a particular input argument can be repeated
multiple times. An example is the include path ({\tt -I}, of the list of
preprocessor definitions {\tt -D}.  Whereas string options must be used when a
single instance is allowed in the argument list. 

Once a flag or option is defined, the framework generates automatically a number
of {\tt static} fields in the \type{utils::CommandLineOptions} class. Exactly
one field per option/flag is provided and the name of the field is the value of
the second argument specified in the XMacro file (that's the reason why names of
command options must be unique). Population of those fields is taken care by the
{\tt utils::Parse(int argc, char* argv[])} method also defined in
\file{utils/cmd\_line\_utils.h}. After the \decl{Parse()} function is invoked,
the content of the static fields of the \type{utils::CommandLineOptions} class
can be accessed to retrieve the value of a specific flag or option. The decision
of saving those information as static fields has been motivated by the fact that
command line arguments should be available and easy to access at any phase of
the compilation process. Therefore, in order to avoid to pass the
\type{utils::CommandLineOptions} object everywhere within the compiler, the
singleton solution was preferred (by most). Having a singleton at this stage may
however interfere with parallelization, therefore the developer must take care
of maintaining consistency. 

\subsubsection{How to Add a New Command Line Option/Flag?}
Add the new definition in the \file{utils/options.def} file specifying all the
required informations (as shown in the previous example).

\subsubsection{How to Access to Command Line Option/Flag's Value?}
The \decl{Parse()} method is invoked in the Insieme main function before
anything is performed. Therefore the value of the input arguments are
immediately available to all the modules of the compiler. An example of how to
access command line arguments is well documented in the main function of the
Insieme compiler in \file{driver/main.cpp}. However an example is the following:

\begin{srcCode}
// Parse input arguments and populate the CommandLineOptions object
utils::Parse(4, { "./a.out", "-I.", "-I/usr/include", "-v=2"}); 

for(const std::string& path : utils::CommandLineOptions::IncludeFiles) {
	std::cout << path << std::endl;
}
// Prints to standard output: ".", "/usr/include"

if (utils::CommandLineOptions::Verbosity>2) {
	std::cout << "Very verbose output" << std::endl;
}
\end{srcCode}

\subsubsection{Traps and ``Non-Standard'' Use Cases}
A known trap is related to availability of input arguments when the compiler is
instantiated within the compiler like in test cases. Because there are no input
arguments to parse, \type{CommandLineOptions}' field will not be correctly
initialized and compiler code which depends on the value of those fields may
stop working. 

There are two ways of correctly dealing with this problem. You can manually
build the arguments for the \decl{Parse(...)} function as shown above.  The
second way is to set the value of \type{CommandLineOptions}' field manually
before invoking the desired compiler module. An example of this solution to the
problem was implemented to provide a \emph{facade} interface to the frontend
(\file{frontend/frontend.h}). 



\subsection{Logging}
\subsection{Compiler}
\subsection{Lua}
\subsection{Others}
\subsubsection{Container Utilities}
\subsubsection{Set Utilities}
\subsubsection{Cache Utilities}
\subsubsection{String Utilities}
\subsubsection{Type Traits}




\section{The Core}
The main obligation of the core module is to provide the data structures
required to represent program code based on the Insieme Parallel Intermediate
Representation (INSPIRE). In addition to the raw data structures, a set of
primitives and utilities to inspect, address, analyse, manipulate and load/store
IR codes are offered to enable a high level interaction with the sometimes rather
low-level elements. Throughout the remaining compiler modules, this
representation is used as an exchange format when dealing with programs.


\subsection{The Core's Contributions [Herbert]}
\subsubsection{Representation}
The compilers intermediate representation is a formal, Turing-complete,
high level programming language specified within the Insieme IR
Language Specification \cite{insieme_ir_spec}. Internally, the various
constructs required to represent programs are modeled using nodes within a DAG 
(see \ref{sec:Compiler.Core.NodesAndManagers}). Each node has a certain type
(e.g. a variable, a for-loop or an array-type) and a list of child-nodes
representing parameters of the individual node types (e.g. value type, loop
body or length or an array). Although virtually representing a tree, the nodes
are physically organized withing a DAG since many sub-trees within a program
representation (e.g. reused struct types or functions) are identical. By using a
DAG structure, these unnecessary redundancy can be avoided by sharing common
sub-trees. However, as a consequence of the sharing, all nodes within the DAG
have to be immutable.

\subsubsection{Construction}
Every data structure needs to be constructed at some point. All nodes of an IR
DAG can be constructed using static factories associated to the corresponding
node types (see \ref{sec:Compiler.Core.NodesAndManagers.Factories}). In many
cases, this is a very cumbersome way of constructing IR codes since various
frequently used constructs involve the construction of several nodes.
Furthermore, proper typing of expressions is required. Therefore, a builder has
been implemented supporting the construction of typical, frequently used IR
fragments. Its internal organization is covered within section
\ref{sec:Compiler.Core.Builder}. However, for more extensive code sections like
a simple loop nest, using the builder still requires a considerable amount of
coding effort making the resulting code unreadable. Also, it is sometimes next
to impossible to deduce from the code what has been constructed. Therefore, an
\textit{IR Parser} converting a string-based input IR code is provided to offer
a convenient mean to build larger IR constructs. For details see
\ref{sec:Compiler.Core.Parser}.

\subsubsection{Addressing}
Nodes within the IR of a program can be addressed using two distinct means. The
simple one is by obtaining a pointer to the corresponding node. However, in some
cases this is not sufficient, since a pointer to a node within a DAG does not
allow to distinguish multiple usages of the same, shared structure. For
instance, a pointer to a variable node is simply referencing this variable.
Since the same node is commonly referenced by any location the variable is
touched by, a pointer does not allow to determine which actual ``position'' in a
program is to be addressed. However, pointers are a cheap way of addressing
elements and in many cases the actual context is not important. If, however, the
context gets important, \textit{NodeAddresses} can be used. Unlike pointers,
addresses model the full path from some root node (not necessarily the global
root node) to an addressed node within the virtual IR tree. Therefore, all the
context information is available. Addresses may for instance be used when aiming
for replacing specific sections of a program -- sections which might be present
in an identical form within the program at different places. More regarding the
implementation of addressing means in the IR is covered within the corresponding
section \ref{sec:Compiler.Core.PointersAndAddresses}. Section
\ref{sec:Compiler.Core.Visitors} lists the various means offered to iterate over
(sub-sections) of an IR tree using \textit{Visitors} while section
\ref{sec:Compiler.Core.Mappers} covers the basic mechanism for ``mutating'' IR
constructs based on Node \textit{Mappers}.

\subsubsection{Extensions}
Not every construct defined by the INSPIRE language is converted into a
specialized IR node. For instance, basic types (integers,
floats, specialized arrays, \ldots) and a list of operations manipulating
values of those types (addition, array accesses, \ldots) are integrated using a
standardised procedure. The implementation of the procedure used for defining
the core language (also known as Lang-Basic) and various, special-purpose
language extensions are covered within section
\ref{sec:Compiler.Core.LangBasic}.

\subsubsection{IO}
Section \ref{sec:Compiler.Core.Printer} describes the implementation of the IR
pretty printer producing ``human readable'' yet not parse-able output intended
for more effective debugging. The IR Dumpers covered within section
\ref{sec:Compiler.Core.Dumpers} however are capable of storing and loading IR
codes using binary or text-based formats.

\subsubsection{Checks}
In addition to the means representing, printing and modifying applications,
utilities enabling to verify the proper composition of IR codes are offered. The
most essential contribution thereby is the type deduction mechanism covered
within section \ref{sec:Compiler.Core.TypeDeduction}. This mechanism is used
to automatically deduce the type of an expression as well as to verify whether a
given type is matching an expression. It is therefore the foundation for a
series of additional semantic constraints imposed on IR codes and verified by
the checks covered within section \ref{sec:Compiler.Core.SemanticChecks}. 

\subsubsection{Analysing}
A program representation is of no big use if it cannot be investigated. Visitors
provide a very flexible mean to do so, yet they are rather low level constructs.
Although a large number of frequently occurring questions (e.g. is a given type
a reference type? What type is it referencing? Is type A a specialization of
type B?) can be implemented using visitors within a view lines of code, their
application results in unnecessary extensive code sections. Therefore,
frequently required analysing utilities are collected and aggregated within this
sub-module and documented within section \ref{sec:Compiler.Core.Analysis}.
Further, section \ref{sec:Compiler.Core.Statistics} covers utilities implemented
for collecting statistical information regarding the size and depth of an IR
tree, sharing parameters, memory consumptions and node-type distributions.

\subsubsection{Manipulation}
As Visitors provide a foundation for all kind of analysis, Mappers are the
primitive used for altering IR codes. However, du to their additional required
flexibility Mappers are even more cumbersome to employ for particular purposes.
Therefore, basic operations like adding, removing or replacing nodes within an
IR, updating a variable type, inlining, outlining and variable substations are
offered to provide easy-to-use manipulation utilities. The corresponding means
are covered within section \ref{sec:Compiler.Core.Manipulation}.

\subsubsection{High-Level Manipulation}
In some cases thinking about IR constructs using more abstract terms is
simplifying the interaction. For instance, when manipulating arithmetic
expressions common operations like the +, -, * and / should be supported.
Therefore an infrastructure enabling the interpretation of IR constructs for
specific domains has been established. For once, this should simplify
interaction. However, it should also provide incentive to represent information
e.g. obtained by analysis using the already present IR constructs. This was, the
IR is naturally extended to represent any kind of value. The benefit is, that
present utilities can be used for the manipulation of this data (e.g. IO
utilities) and the results can be easily integrated into the output code. The
corresponding means are covered within section \ref{sec:Compiler.Core.Encoding}.
Furthermore, section \ref{sec:Compiler.Core.Arithmetic} covers the the
application of this concepts regarding (partial) arithmetic formulas
encountered frequently within numerical applications.

\subsection{The Nodes and their Managers}
\label{sec:Compiler.Core.NodesAndManagers}
INSPIRE, the Insieme IR, is a formal language designed for modeling all aspects
of a program using a compact, uniform representation. This covers types,
statement and functions as well as expressions (=values).

\subsubsection{An Overview}
An essential obligation of the compiler core is to provide the data structure
required for representing IR codes within a C++ application. Due to the
hierarchical structure of INSPIRE, every code fragment can be modeled as a
simple tree -- where each node has a type and a (potentially) variable length
child list. However, implementing such a representation in a straight-forward
way would require a large number of redundant sub-trees since. For instance,
frequently used IR trees like the one representing the simple integer type --
which due to modeling reasons require at least 3 nodes, one for the integer
family, one for the precision and another to combine those two properties into a
(generic) type -- would be instantiated several thousand times within a medium
size code segment since every integer expression has a child referencing its
type. To circumvent this unnecessary data and memory management overhead the
IR was implemented based on a directed, acyclic graph (DAG). Hence, virtually
the IR represents a tree while physically being formed by a DAG. Equivalent
sub-trees are automatically referencing identical sub-tree instances as long as
those are maintained within the same \texttt{NodeManager}.

\paragraph{Nodes} \index{IR!Node} The elements within the IR DAG (virtual IR
tree) are simple refereed to as \textit{Nodes}. Nodes follow the following
simple structure
\begin{align}
	\label{eq:Compiler.Core.NodeDef}
	\mathcal{N} &::= \mathcal{V}\;|\;\mathcal{T}(\mathcal{N}^*) \\
	\mathcal{V} &::= \mathbb{Z}\;|\;\text{String}
\end{align}
where $\mathcal{N}$ is the set of all nodes, $\mathcal{T}$ the set of all node
types, $\mathbb{Z}$ the number of integers and $\text{String}$ the set of
character sequences. Hence, every node is either a value or a combination of a
node type and a list of child nodes. The node type is thereby defining the
interpretation of the various child nodes. Consequently, not every
type/child-node list combination is a valid combination. The valid combinations
are enforced by the factory functions.

\paragraph{Factories} \index{IR!Factory} \label{sec:Compiler.Core.NodesAndManagers.Factories}
The implementation of nodes is only offering private constructors. To create a
node, static factory methods of the individual classes have to used. The
factories are enforcing the proper composition of the child-node lists. Further,
they are ensuring that all Node instances are instantiated on the heap -- no IR
node is allowed to be located on the stack due to the required, explicit memory
management. To manage the life-cycles of nodes and other inter-node constraints,
\texttt{NodeManager}s are required.

\paragraph{NodeManager} \index{IR!NodeManager} The life cycle of Nodes forming
DAGs are managed by \texttt{NodeManager}s. Nodes within a \texttt{NodeManager} a kept alive as
long as the node manager is alive. As soon as the manager is destructed, all
managed nodes are destroyed as well. From this, a simple rule follows: all child
nodes of a node have to be manage by the same \texttt{NodeManager} as the parent node.
\texttt{NodeManager}s should be used for conducting local operations involving the
construction of a large number of nodes. By creating a local manager, copying in
the input IR, conducting the manipulations and copying back the result into the
manager of the input, all temporal nodes will be automatically destroyed after
leafing the local scope.

\index{IR!NodeManager chaining}
To eliminate the overhead of copying large IR DAG structures between a given and
a local temporary \texttt{NodeManager}, managers can be chained to form hierarchies of
nested local scopes. By passing a \texttt{NodeManager} instance to the constructor of a
\texttt{NodeManager}, the newly created manager will ``inherit'' all node instances from
the handed in manager instance. Therefore, all the nodes present within the
original manager will not be copied into the temporary node manager. This
mechanism therefore provides a cheap way to define nested, hierarchical scopes.
Therefore, actually, all child nodes of a node have to be manage by the same
\textit{or a higher level} \texttt{NodeManager} as the parent node.

Beside the life-cycle management \texttt{NodeManager} are also realizing the implicit
node sharing. References to equivalent nodes managed by the same \texttt{NodeManager} are
guaranteed to point to the same object in memory. Therefore, each factory
function requests a reference to the \texttt{NodeManager} the resulting node should be
managed by. Before actually constructing the resulting Node it is searched
within the Manager. If it is already present, the existing instance is returned.
Otherwise a new one is created and registered.

In the implementation the factory is actually creating a temporary instance on
the stack -- which the factory as a member function of the node class is
allowed to do -- and searches for an identical instance within the manager.
Since the manager is based on a HashSet this is a code-efficient and save way of
searching for duplicates. If a duplicate exists, the duplicate inside the
manager is returned. Otherwise the local copy is cloned to the heap by the
manager, registered internally and returned. In both cases, the local instance
created by the factory is destroyed.

IR nodes can not be allocated on the stack. This is enforced by the
correspondingly overridden new / delete operators. The only exception are the
temporary allocated IR node instances created within factory functions to be
used for the look-up within the \texttt{NodeManager}. Hence, every IR node
encountered outside the factories is located on the heap.


\subsubsection{Classes and Files}
The following important classes are forming the foundation of the IR
constructs:
\begin{itemize}
  \item Node \ldots the base class of all IR nodes
  \item NodePointer \ldots Pointer type used to reference nodes managed by a
  NodeManager
  \item NodeType \ldots an enumeration of all node types, e.g.
  \textit{NT\_CallExpr}
  \item NodeCategory \ldots an enumeration of all node categories, e.g.
  \textit{NC\_Expression}
  \item NodeManager \ldots the manager used for life-cycle and node-sharing
  management
  \item NodeAccessor \dots a generic base class for classes accessing node
  properties including nodes themself, NodePointers and NodeAddresses (see
  section \ref{sec:Compiler.Core.PointersAndAddresses})
  \item NodeAnnotation \dots the base class for all annotations attachable to
  IR nodes
\end{itemize}
Additionally a number of externally invisible utility classes and type trait
structs defining MPL constants and type relations are used. Their scope, purpose
and interaction with other parts of the code should be fairly obvious such
that the in-source documentation should suffice. All of the listed classes are
defined within the \lstinline|insieme::core| namespace.

Additionally, each NodeType has its own implementation type (=class), forming a
simple is-a inheritance hierarchy. A list of all IR nodes and their direct
parent classes can be found within the \textit{ir\_nodes.def} X-macro file.

The following source and header files are contributing to the IR node
definition, where ``xy.*'' corresponds to the header file ``xy.h'' as well as the
implementation file ``xy.cpp'':

\begin{itemize}
  \item \textit{forward\_decls.h} \ldots covers a list of frequently used type
  declarations including all specialized NodePointer and NodeAddresses
  \item \textit{ir\_node.*}\ldots defines the Node base type, the \texttt{NodeManager}
  and some generic utility classes like the ListNodeHelper (an accessor for
  nodes containing a homogeneous list of child nodes) and the FixedSizeHelper
  (an accessor for nodes without this property)
  \item \textit{ir\_nodes.def} \ldots an X-macro file defining the set of all
  nodes and their relations
  \item \textit{ir\_node\_accessor.*} \dots defines the NodeAccessor base class
  \item \textit{ir\_node\_types.*} \ldots defines the NodeType and NodeCategory
  enumerations based on \textit{ir\_nodes.def}
  \item \textit{ir\_values.*} \dots defines node types representing atomic
  values including integers and strings
  \item \textit{ir\_int\_type\_param.*} \ldots defines all node types
  representing integer parameters for IR types
  \item \textit{ir\_types.*} \ldots defines all node types to represent IR types
  \item \textit{ir\_expressions.*} \ldots defines all node types to represent IR
  expressions
  \item \textit{ir\_statements.*} \ldots defines all node types to represent IR
  statements
  \item \textit{ir\_program.*} \ldots defines the node used to represent an IR
  program (typically the root node of a full program)
\end{itemize}


\subsubsection{Nodes and Accessors} \index{IR!Accessor}
The implementation of the IR nodes is a quite regular task. The \textit{Node}
base type is managing the node type as well as the child-node list respectively
the node values, which are the only members defining a node's identity (see
grammar rule \ref{eq:Compiler.Core.NodeDef}). Therefore, functions like the
equals operator or the hash computation required for maintaining IR nodes
within the \texttt{NodeManager} (which is based on an HashSet) have been implemented
as part of the base class. However, for the individual Nodes observer functions
(getters) for the various members have to be implemented -- which turned out to
be a more subtle story.

As has been mentioned before (and will be covered in more detail within the
following section), nodes within the IR DAG can be addressed using NodePointers
and NodeAddresses. For instance, let \lstinline|call|
be an instance of a \lstinline|CallExprPtr| representing a pointer to a
\lstinline|CallExpr| node.
When accessing the type of this expression using \lstinline|call->getType()| the
user is expecting a \lstinline|TypePtr| describing the type of the value
represented by the given call-expression. When \lstinline|call| is a
\lstinline|CallExprAddress| the expression \lstinline|call->getType()| is
expected to return an address of type \lstinline|TypeAddress| referencing the
same type node as in the previous case, however, addressed via an address
extending the \lstinline|call| address by an additional level.

To realize this ``intuitive'' behaviour, a lot of error prone, manual
implementation effort would be necessary -- since every Pointer and Address Type
has to implement all the observer/getter functions -- or, as in our case, a
slightly more sophisticated architecture and a little bit of C++ template
magic.

The basic idea is to separate the data objects from an abstract implementation
of the observer functions. The data is stored in the node instances (almost
everything in the basic Node type) and the access functions are defined in a
generic way within the NodeAccessors. The goal of the design is to implement
relevant functionality (e.g. the getType() method of the CallExpr Node) at a
single place although still effecting all the involved parts (Node, Pointer,
Address, \ldots). Further, all implementations should preserve type information
as far as possible. For instance, requesting a type should return a TypePtr /
TypeAddress and not a more generic NodePtr / NodeAddress since the latter would
require the user to apply frequent casts. Finally, by offering utilities class
and macro definitions, the amount of effort required to add / modify node
implementations -- and therefore the likelihood of introducing an error --
should be minimized.

Figure \ref{fig:Compiler.Core.Classes.ReturnStmt} illustrates the concept based
on a simple node type -- the ReturnStmt.
\begin{figure}[tb]
	\centering
	%\includegraphics[width=\textwidth]{compiler/core/class_hierarchy_of_return_stmt.pdf}
	\includegraphics[width=\textwidth/4*3, trim=0 50 670 0, clip]{compiler/core/class_hierarchy_of_return_stmt.pdf}
	\includegraphics[width=\textwidth/4*3, trim=670 0 0 0, clip]{compiler/core/class_hierarchy_of_return_stmt.pdf}
	\caption{Relations between Classes contributing to the ReturnStmt}
	\label{fig:Compiler.Core.Classes.ReturnStmt}
\end{figure}
The blue boxes mark types being exposed to the user. The gray boxes are various
instantiations of generic types contributing the required functionality. The
following concrete classes are contributing to the architecture:
\begin{itemize}
  \item Node \ldots the generic base type
  \item Statement \ldots the common base type of all statements extending the
  node. This type inheritance allows Statements to be treated like any
  other Node.
  \item ReturnStmt \ldots the class implementing a return statement in the IR.
  IT is a extending the Statement type. It is only defining the type and a
  factory function (see code documentation), yet it does not define observer /
  getter methods.
  \item NodeAccessor \ldots a generic type defining all observer /
  getter methods applicable to nodes in general. The template is exploiting
  static dispatching for addressing requests to the actual instance derived from
  the generic base class. This way, overheads in memory requirements and
  execution time should be avoided. The second template parameter defines the
  reference type returned by involved getters. For more details on the
  parameters see the source code documentation \cite{insieme_source_doc}.
  \item StatementAccessor \ldots a generic accessor extending the basic
  NodeAccessor by all observer functions offered by any statement
  \item ReturnStmtAccessor \ldots a generic accessor extending the
  statement accessor by the extra functions offered by ReturnStmt nodes.
  \item ReturnStmtPtr \ldots the type of pointer used to reference ReturnStmt
  nodes. It uses the same hierarchy of generic types to obtain the required
  functionality as the node implementation itself does. However, it used
  different types to instantiate the offered generic parameters. Hence, every
  observer function only needs to be implemented once -- in a generic context.
  \item ReturnStmtAddress \ldots the address based counterpart of the
  ReturnStmtPtr
  \item FixedSizeNodeHelper \ldots a generic variadic template utility class
  implementing type save accesses to members of the child list -- which forms
  the foundation for all observer functions defined within the Accessors.
\end{itemize}


In the implementation two types of nodes are distinguished. The first, the
\textit{FixedSizeNode} \index{IR!FixedSizeNode} type covers nodes having a fixed
size list of arbitrarily typed child nodes (like a tupel). For instance, the
ReturnStmt is a FixedSizeNode having a single child node -- the expression
computing the value to be returned. The second type is the \textit{ListNode}
\index{IR!ListNode}.
In addition to a fixed set of child nodes, ListNodes may have a variable length list
of child nodes of a homogeneous type. An example is the \textit{CallExpr}. It
child list has the shape
$$TypePtr, ExpressionPtr, ExpressionPtr^*$$
where the type is the type of the value represented by the call expression, the
first expression corresponds to the function to be called and the remaining
expressions to the arguments to be passed to the function. For ListNodes the
\lstinline|FixedSizeNodeHelper| is replaced by the \lstinline|ListNodeHelper|
and an additional layer of a \lstinline|ListNodeAccessor<>| class is introduced
at the bottom of the accessor hierarchy to support type-safe accesses to the
tailing node list. Figure \ref{fig:Compiler.Core.Classes.CallExpr} illustrates
the resulting class hierarchy of the CallExpr.
\begin{figure}[!tb]
	\centering
	%\includegraphics[width=\textwidth]{compiler/core/class_hierarchy_of_call_expr.pdf}
	\includegraphics[width=\textwidth/4*3, trim=0 50 650 0, clip]{compiler/core/class_hierarchy_of_call_expr.pdf}
	\includegraphics[width=\textwidth/4*3, trim=650 0 0 0, clip]{compiler/core/class_hierarchy_of_call_expr.pdf}
	\caption{Relations between Classes contributing to the CallExpr}
	\label{fig:Compiler.Core.Classes.CallExpr}
\end{figure}
Note: although the CallExprPtr type inherits from 5 classes it is not virtual.
It also just requires the size of a single pointer in memory and does not
introduce any overhead for construction / destruction except of setting the
pointer-member field. This light-weight property is especially desirable
when considering the fact that instances of NodePointers are created and
destroyed billions of times when running compiler code.

\subsubsection{Macros}
The restrictions imposed on nodes makes the implementation of the actual
instances a quite mechanical task -- and is therefore left to marcos. Within
\textit{ir\_node.h} the following Macros are defined to generate node Accessors
and Node Implementations: 
\begin{itemize}
  \item IR\_NODE \ldots creates the head of a Node
  type definition including classes to be extended and a set of required constructors and a default
  factory. This Macro should be used to define new Node classes and extend those
  by static factory methods. 
  \item IR\_NODE\_ACCESSOR \ldots creates
  the head of a NodeAccessor definition for a FixedSizeNode including all
  dependencies and some member type definitions. The macro accepts a variable
  length argument list which has to be instantiated by the list of types
  expected to form the child list of the corresponding node.
  \item IR\_LIST\_NODE\_ACCESSOR \ldots the same as the IR\_NODE\_ACCESSOR
  macro, yet generating a NodeAccessor definition header for ListNodes.
\end{itemize}
However, those might only be used for creating
concrete nodes (nodes forming leafs of the type hierarchy, e.g. CallExpr), not
for abstract nodes within the type hierarchy (e.g. Statement, Expression,
\ldots). Those need to be implemented manually. For an extensive set of examples
on their application readers are referred to the existing node implementations.

\subsubsection{The NodeType Enumeration}
The X-macro file \textit{ir\_nodes.def} defines the list of all existing node
types and their direct inheritance relation ship. It is used at various places
for statically generating codes depending on this information using the C
pre-processor. An example case is the definition of the NodeType enumeration
within \textit{ir\_node\_types.h}. 

Every node instance is exposing its type via the \lstinline|getNodeType()|
method (defined within the generic NodeAccessor). This function can therefore be
used to determine the type of a node.


\subsubsection{Special Algorithms}
Within this subsection a view specific algorithms involved in the managing of
nodes will be discussed. 

\paragraph{Implicit Node Sharing} \index{IR!Node Sharing}
The virtual IR trees are internally maintained as a DAG by sharing identical
sub-trees. To realize this, a mean allowing to determine whether a certain node
instance already exists is required. This is among the central obligations
of the \texttt{NodeManager}s (beside the life-cycle management).

The InstanceManager, a generic utility class the \texttt{NodeManager} is based on, is
maintaining a set of all nodes currently ``alive'' within its domain. This set
is a hash based data structure. Therefore, every element within the set has to
offer a hash value and an equality function. Both requirements are satisfied by
the Node base class (which is possible do to the constraints defined by
\ref{eq:Compiler.Core.NodeDef}).

Whenever a Node is created, the following procedure is conducted:
\begin{enumerate}
  \item The Type and the Child-Node List is obtained (e.g. via an argument of
  the factory function)
  \item A local instance of the resulting node is created on the stack
  \item It is passed to the manager to obtain the ``master copy'' of
  requested node
  \item The manager computes the nodes hash value (by using one of its methods)
  to perform a look-up; the look-up first checks base-managers recursively (if
  chained) before checking the locally maintained set
  \item If present, the manager returns a reference to the obtained master copy
  \item If not present, a new master copy is created by cloning the handed in
  temporary node and a reference to the new node is returned
  \item The temporary copy on the stack is destroyed 
\end{enumerate}

In the concrete case of a ReturnStmt all those steps are conducted within one of
its factory functions using the following line of code:
\begin{lstlisting}
	return manager.get(ReturnStmt(expression));
\end{lstlisting}

In the end, although the mechanism beneath is complex, the interface to the user
is kept as simple as possible. The node-sharing happens automatically.

\paragraph{Hashing} \index{IR!Hashing}
The hashing of Nodes is implemented within the static member function
\lstinline|hashNodes| of the Node class and the \lstinline|hash| functions
within the \textit{ir\_node.cpp}. All of them are simply combining the hash
values of the child nodes with the NodeType using a boost utility function to
obtain a proper hash code. Therefore, the hash value of a node recursively
covers the entire virtual sub-tree rooted by the given node.

Since hashes are depending on the entire sub-tree, computing those can become
expensive. In fact, the recursive computation grows exponential with the height
of the IR tree. Fortunately, all nodes are immutable. Therefore, hashes can be
cached and computed once during construction -- by only accessing the already
cached hash codes of the child nodes. This way, the computation of the hash is
reduced to a local operation, hence $\mathcal{O}(1)$;

\paragraph{Equality} \index{IR!Equality Classes} 
For the look-up in the HashSet Nodes need to be tested for equality. If two
nodes are equivalent, this is quite easy. With a probability bordering on
certainty their hash values will be different. If not, node types or the
immediate child nodes are likely to be different. Hence, inequality can be
quickly (almost always locally) decided.

However, in case two nodes (and hence the represented sub-trees) are equivalent,
this cannot be decided as effectively with the provided means. Just because hash
values are identical does not allow to conclude that the corresponding trees are
equal. To make sure two trees are identical, every node within one tree has to
be compared to every node within the other tree. Hence, deciding whether two
trees are equivalent is an expensive operation -- considering the fact that
trees may consist of more than 10 million (virtual) nodes for moderate sized
codes.

To fix this problem, equivalence classes of nodes identified by simple integer
IDs are formed dynamically to cache comparison results. The mechanism is
implemented by the == operator of the Node class. Each node is created using an
equality class ID of 0 (hence, no class fixed yet). If it is ever compared to
another node and determined to be equivalent, the equality ID of the other node
is copied. If the other ID is also 0, a fresh equality ID is assigned to both
nodes. Future comparisons start by comparing the equality ID.
If both are not 0 and equal it can be immediately deduced that nodes are equal. If both are
not 0 and distinct, the comparison has to be continued recursively (they still
might be equal, just their equality classes never ``met'' so far). Finally, in
case two nodes are determined to be equivalent although their equality ID is set
and distinct, the larger equality ID will be replaced by the smaller one. This
way, equality classes are merging over time.

The locally stored integer allows to determine equality effectively using
locally available information. Even in the cases the local nodes does not have a
set equality ID, it is likely the child nodes have. Amortised, this should
result in a complexity of $\mathcal{O}(1)$ for determining whether two IR trees
are structurally identical.


  

\subsubsection{HowTo}
Finally, to conclude the documentation of the Node implementation some example
of common steps should be documented:

 \paragraph{Add a new Node} When ever adding a step, follow the following
 procedure:
 
\begin{enumerate}
  \item Do you really need a new node type or would a type, function or
  combination of existing nodes be sufficient?
  
  \item So, you're serious, you really need a new one? If not, just stop here.
  
  \item Register the new Node within the type hierarchy within
  \textit{ir\_nodes.def}. This will automatically create an entry in the
  NodeType enumeration, add support within visitors, IR statistics and printers.
  
  \item Determine whether your new node is a FixedSizeNode or a ListNode.
  
  \item Pick a file to implement your new node. In the header file you have to
  define the accessor and the node class itself. Use the corresponding macros
  defined within \textit{ir\_node.h}. When defining the accessor, fix the list
  of child node types.
  
  \item Add a list of accessor functions to the accessor using the
  IR\_NODE\_PROPERTY macro. Each macro is mapping a type and name to a certain
  location within the child-node list.

  \item If necessary you can add derived accessors (accessing nodes of
  child-nodes). For examples see the implementation of the ForStmtAccessor.
  
  \item Add required factory functions to the body of the node class definition.
  
  \item Rerun the cmake-script to trigger the IR builder generator (see
  \ref{sec:Compiler.Core.Builder} for details)
  
  \item Test your new node. You may extend one of the existing test cases.
  Essentially, would should be at least (!) done is to create an instance -- or
  more if complex combination of child lists are possible -- and run the
  \lstinline|basicNodeTests| implemented within \textit{ir\_node\_test.inc}
  located in the test folder of the core project.
\end{enumerate}
 
 
\paragraph{Add a Member Field to a Node Type}
To add a new property to a node, process the following steps:
\begin{enumerate}
  \item Determine whether the new field requires a new entry in the child-list
  or whether it is a derived one.
  
  \item If derived: use the examples included within the ForStmt
  (for deriving nodes) or the LambdaExpr (for derived non-node
  values) implementations as an example
  
  \item If a new filed:
	\begin{itemize}
	  \item add the type of the new field to the child-type list
  	  within the corresponding NodeAccessor
  	  
  	  \item add a access method using the IR\_NODE\_PROPERTY macro

  	  \item if the new child node has not been appended to the end, fix the
  	  indexes of the remaining methods

  	  \item update the factory function, which will update the builder when
  	  running cmake the next time

  	  \item fix all depending utilities like the pretty printer, the backend or
  	  any code using the factory function (if not backward compatible)
	\end{itemize}
  
  \item Extend the existing test cases to cover the new node
\end{enumerate}

\paragraph{Add a Method to a Node Type} 
To add new method to a node, you should first think whether your desired
operation is of general interest -- hence, whether it has a wide enough
relevance to justify adding it to the node itself instead of realizing it
as a function within some external utility header. The node implementation
should be kept clean of use-case specific codes.

If you decide that your method is important / useful / convenient enough, you
should add it using the same procedure as has been described for derived members
within the previous HowTo. As an example you can have a look at the
\lstinline|isRecursive()| method of the \lstinline|LambdaExpr| node.


\paragraph{Migrate Codes between Node Managers}
When working with multiple node managers frequently the question regarding how
to migrate an IR DAG from one node manager to another. The answer is simple:
just ask the target manager for a reference to the corresponding IR structure.

For instance, let \lstinline|code| be a reference (NodePtr) to the root of some
IR structure. Let \lstinline|targetMgr| be the manager you want to obtain a copy
from it. Simply use 
\begin{lstlisting}
	auto copy = targetMgr.get(code);
\end{lstlisting}
to obtain the corresponding copy within the target manager. If your
\lstinline|nodeAddr| is of type \lstinline|NodeAddress| (or any other node
address) you can use 
\begin{lstlisting}
	auto copy = nodeAddr.cloneTo(targetMgr);
\end{lstlisting}
to obtain a lone of the address within the given target manager. Unlike the
pointer variant, this does not only clone the addressed node to the target
manager but also all the nodes along the path described by the address,
in particular including the root node.

\paragraph{Use Node Managers to form Local Scopes}
When entering a code section performing a heavy node manipulations,
hence generating a large number of temporary nodes it is recommended to use some
local manager to limit the life-cycles of temporary nodes. This mechanism can be
considered as some simple mechanism of garbage collection.

The following code example demonstrates how to use \texttt{NodeManager} chaining:
\begin{lstlisting}
	NodePtr process(const NodePtr& input) {
		// get reference to outer node manager
		NodeManager& outer = input->getNodeManager();
		// create local node manager
		NodeManager local(outer);
		// work with local manager
		NodePtr res = doSomething(input, local);
		// return result - ATTENTION: within the right node manager
		return outer.get(res);
	}
\end{lstlisting}
Some functions do not allow to pass a \texttt{NodeManager} as an argument. For those,
chained managers can not help to contain temporary node instances since they are
using the \texttt{NodeManager} managing the input node. Note, passing
\lstinline|local.get(input)| as an argument does not help, since it is just the
same as passing \lstinline|input| (inherited from the outer \texttt{NodeManager}).

In the core, most operations offer the possibility to pass the \texttt{NodeManager} to be
used as an argument. Otherwise, to isolated the creation of temporary nodes, an
independent \texttt{NodeManager} instance can be used as follows:
\begin{lstlisting}
	NodePtr process(const NodePtr& input) {
		// get reference to outer node manager
		NodeManager& outer = input->getNodeManager();
		// create local node manager
		NodeManager local;
		// work with local manager
		NodePtr res = doSomething(local.get(input);
		// return result - ATTENTION: within the right node manager
		return outer.get(res);
	}
\end{lstlisting}
However, in this case all nodes referenced (transitively) by the input node will
have to be copied to the new node manager. On the other hand, the total
isolation even enables to work on the various copies in parallel, as will be
covered within the following section.

\paragraph{Use multiple Node Managers in Parallel Contexts}
Although generally immutable, the managing of nodes within the manager and the
handling of annotations is not synchronized. \note{This is something which
really should be fixed in the future!} Therefore, when operating on IRs in
parallel including the creation of nodes, each thread has to work on its own
copy using a local (private) non-chained node manager. 

Therefore, just create a local \texttt{NodeManager} and clone the IR to be processed to
your local manager. Further on you can run your operations in parallel on the
individual clones. Attention: if you want to return a value, you have to copy
the result back into the original \texttt{NodeManager} -- which should happen in a
synchronized way.

Some operations operating on IR nodes require the generation of fresh IDs (e.g.
variable IDs). Those IDs are managed by the \texttt{NodeManager}. After creating a local
manager, the next ID will be 1 again -- since none has been assigned yet.
Further, cloning IR structures into the manager will not change this
circumstance. In rare cases it can happen this way that presumably unique IDs
are used for new nodes which further on produce collisions since the same IDs
have been used before.\note{This is also something which should be fixed
soon. (I bet this note will be here forever)}

To prevent this from happening, after creating your private \texttt{NodeManager} the
fresh-value ID should be seeded. This can be conducted using the following code
snippet (also includes the parallel node manager creation):
\begin{lstlisting}
	core::StatementAddress trg = someSource(...);
	
	#pragma omp parallel
	{
		core::NodeManager local;
		core::StatementAddress localTargetCopy;

		#pragma omp critical
		{
			local.setNextFreshID(trg->getNodeManager().getFreshID());
			localTargetCopy = trg.cloneTo(local);
		}
	
		// run parallel operations

	}
	// temporaries automatically cleaned up
\end{lstlisting}


\begin{comment}
The main obligation of the core is to provide the data structures to model 

Sections:
\begin{itemize}
  \item General idea of the DAG and the sharing
  \item Involved classes and files and their relationship
  \item Implementation details of the classes
  \item The various node types
  \item Special Algorithms - hashing, equality
  \item HOW-TOs
\end{itemize}


To be covered:
\begin{itemize}
  \item Purpose of Nodes - distinction to the IR Spec
  \item Fact that nodes form a DAG
  \item Node Identity specified by type + children or value (little formally)
  \item Node Manager concept for memory management and sharing
  \item Handling the node manager in parallel environments
  \item Implementation of nodes (macros and generics)
  \item List of involved files (nodes.def, ir\_nodes.h, ir\_types.h)
  \item NodeType, NodeCategory, Nodes, Values, Types, Expressions, Statements,
  Support, Program, Accessors, \ldots + relations
  \item How to add new nodes
  \item How to add a method to a node
  \item How to add a field to a node (when to do so - e.g. not if it is
  something identity critical)
  \item Hashing and equality determination.
\end{itemize}

\subsubsection{Types}
\subsubsection{Statements}
\subsubsection{Expressions}
\subsubsection{Support}
\end{comment}

\subsection{Of Pointers and Addresses}
\label{sec:Compiler.Core.PointersAndAddresses}

The core offers two means to address IR DAG structures. On the one hand,
\textit{Node Pointers} may be used to reference a single node within the DAG.
Since every node is the root of a virtual IR tree, pointers can just as well be
used to refer to IR code (sub-)fragments. \textit{Node Addresses} on the other
hand are addressing nodes being located relative to a certain ``root'' node. For
instance, an address may address the 2nd child of the 3rd child of the 1st child
of a \textit{ForStmt}. The ``root'' not necessarily need to be the root of the
full program. Relative addresses are fine too.


While pointers are sufficient for almost all applications, in some cases
addresses are required to resolve ambiguities caused by the implicit node
sharing. For instance, consider the IR structure used to model the simple
expression \lstinline|1+1|. There is a node representing the 1, including
another node for its type, and a node representing the + operation. The latter
has 2 child node references, the operands, which in this case will both point to
the node representing the 1. Imagine you want to refer to the first operand
only, for instance, to replace it by a node representing 2. Using a pointer does
not help, since a pointer referencing the first operand is also referencing the
second argument (right, due to the implicit node sharing). To resolve this
problem, addresses are offered. By enabling the user to address a node by
following a path within the ``virtual'' IR tree, ambiguities can be resolved.


Table \ref{tab:Compiler.Core.ptr.a.addr} summarizes the differences and
should provide some hints on when to use what.

\begin{threeparttable}[1h]
	\centering
	\begin{tabular}{c|c|c}
		\textbf{Aspect} & \textbf{Pointer} & \textbf{Address} \\
		\hline \hline
		speed & fast as a plain pointer    & slow \\
		space & \lstinline|sizeof(int*)|   & \lstinline|3 * sizeof(int*)| \\
		thread save & yes & yes/no\tnote{1} \\
		\hline
		equality & address of targeted node & root address + rel. path\tnote{2}
		\\
		\hline
		access node and sub-nodes  & yes & yes \\
		implicit up-cast & yes & yes \\
		typesafe casts & yes & yes \\
		typesafe navigation & yes & yes \\
		integrated in visitor & yes & yes \\
		navigation to parent & no\tnote{3} & yes, up to root\tnote{4} \\
		call context information & no & yes, up to root \\
		annotateable & no & no \\
		\hline
		\multicolumn{3}{c}{} \\
	\end{tabular}
	\begin{tablenotes}
		\item[1] Thread save if constructed directly or using non-shared source
		address
		\item[2] Note that different addresses to the same shared nodes are not
		equivalent
		\item[3] There is no well defined parent for a shared node. You
		should consider the faster possibility of stopping a decent one level earlier
		using pointers instead of using addresses to get back one step
		\item[4]According to the unique list of ancestors specified by the address
	\end{tablenotes}
	\caption{Properties of Pointers and Addresses}
	\label{tab:Compiler.Core.ptr.a.addr}
\end{threeparttable}


\subsubsection{Implementation}
One of the major design goals obeyed by the implementation of pointers and
addresses is to allow provide equivalent syntactical interfaces such that
pointers and addresses can be used interchangeable within generic
implementations (e.g. the visitors \ref{sec:Compiler.Core.Visitors} or the
pattern matcher \ref{sec:Compiler.Transform.Pattern}).

Both, pointers and addresses are generic classes. The generic parameter is
thereby defining the type of node they are referring to. For instance, a
\lstinline|Pointer<const CallExpr>| is referencing a constant call expression
node. However, since using those extensive generic type names is rather
cumbersome, aliases are defined for each node type within
\textit{forward\_decls.h}. The type \texttt{CallExprPtr} can therefore be used
as a replacement for the full name. The address counterpart would be
\texttt{CallExprAddress}.


\paragraph{Files}
The following files are containing the definitions of pointer and address
related constructs and utilities:
\begin{itemize}
  \item \textit{core/ir\_pointer.h} \ldots implementation of the generic
  \texttt{Pointer} class and operators including static and dynamic casts
  \item \textit{core/ir\_address.h} \ldots implementation of the generic
  \texttt{Pointer} class and operators including static and dynamic casts
  \item \textit{core/forward\_decls.h} \ldots type definitions of specialized
  \texttt{Pointers} and \texttt{Addresses} using the \textit{nodes.def} X-macro
  file; additional abbreviations of frequently used types (e.g.
  \texttt{NodeList})
  \item \textit{core/ir\_node\_accessor.h} \ldots defines a generic base class
  for node accessor implementations as well as partial template specializations
  for pointer and address based variations
  \item \textit{utils/pointer.h} \ldots implements the base class the IR
  pointer is derived from as well as a set of operations
  \item \textit{utils/set\_utils.h} \ldots defines a hash-based generic set
  implementation (\texttt{PointerSet}) capable of managing pointers as elements;
  it covers plain as well as smart pointers
  \item \textit{utils/map\_utils.h} \ldots defines a hash-based generic map
  implementation (\texttt{PointerMap}) capable of managing pointers as elements;
  it covers plain as well as smart pointers
\end{itemize}

\paragraph{Accessors}
As has been covered within the node section
\ref{sec:Compiler.Core.NodesAndManagers}, the actual accessor functions allowing
to investigate IR nodes are defined within separated, generic Accessor classes.
These classes are also inherited by the generic \texttt{Pointer} and
\texttt{Address} classes. This way, all accessor functions are inherited.
However, the generic nature fo the accessor infrastructure allows the pointer
class to return pointer types upon member accesses while accessor functions
inherited by the addresses class return extended versions of them self,
referencing to the corresponding child nodes.

Note, the \lstinline|->| operator of pointers and addresses are overloaded to
return pointers to the pointer / address itself. Since the pointer / address is
inheriting the access functions forwarding requests to the actual node, this is
what is according to the users intention.

\subsubsection{HowTo}
Finally, some common problems and solutions related to Pointers and Addresses.

\paragraph{How do I cast a Pointer / Address?}
When casting pointers, up- and downcasting has to be distinguished. Upcasting is
supported automatically by pointers and addresses. Hence, in the following
situtations casts are handled implicitly:
\begin{lstlisting}
	StatementAddress a = someSource(..);
	// here a is automatically up-casted
	NodeAddress b = a;
	// also here
	b = a;
\end{lstlisting}

However, since the down-cast is not generally safe, down-casts are note
conducted implicitly. As for plain and smart pointers, static and dynamic casts
are offered. Static casts are faster, yet require the programmer to ensure that
the cast is valid. dynamic casts are slower since they are verifying the correct
type during runtime. Note: in debug-mode, the current implementation of the
static casts is verifying the proper typing as well. If incorrect, an assertion
is triggered.

An up-cast can be conducted as follows:
\begin{lstlisting}
	NodePtr a = someSource(..);
	StatementPtr b = static_pointer_cast<StatementPtr>(a);
	StatementPtr c = dynamic_pointer_cast<StatementPtr>(a);

	NodeAddress d = someSource(..);
	StatementAddress e = static_address_cast<StatementAddress>(d);
	StatementAddress f = dynamic_address_cast<StatementAddress>(d);
\end{lstlisting}

Be aware, while the static cast is returning a re-typed reference to the
original pointer or address, the dynamic cast returns a fresh, re-typed copy.

Since in many cases, where the type is clear, the syntax of the static cast is
quite extensive, both, pointers and addresses offer a short-cut. The \texttt{as}
member function can be used to quickly conduct a static cast.

\begin{lstlisting}
	StatementPtr a = someSource(..).as<StatementPtr>();
	StatementAddress b = someSource(..).as<StatementAddress>();
\end{lstlisting}

\paragraph{How do I check for Null?}
Pointer and Addresses can be Null (not referencing any element). They are also
bool-convertible. If they are Null, the result will be \lstinline|false|,
otherwise \lstinline|true|. Hence, any pointer / address can be used as a
boolean expression to test whether they are referencing a Null value.

\paragraph{How do I obtain an Address from a Pointer?}
A pointer can be converted into an Address using the corresponding construction.
To avoid accidental conversion, the constructor is marked \lstinline|explicit|,
hence has to be invoked manually. Example:
\begin{lstlisting}
	ExpressionPtr a = someSource(..);
	// implicit
	StatementAddress b(a);
	// explicit
	b = StatementAddress(a);
\end{lstlisting}
Up-casts are thereby supported. The resulting address will use the given pointer
as its root node. Further, the length of the address is 0, hence, it is
addressing the root node itself.

\paragraph{How do I obtain a Pointer from an Address?}
The conversion from an address to a pointer referencing its referenced node is
implicit. Hence, a simple assignment is sufficient.
\begin{lstlisting}
	ExpressionAddress a = someSource(..);
	StatementPtr b = a;
\end{lstlisting}
Alternatively, the member function \lstinline|getAddressedNode()| can be used.
To obtain a pointer to one of the parent nodes \lstinline|getParentNode()| can
be used. To obtain the address of a parent node, use
\lstinline|getParentAddress()|.

\paragraph{How do I create Pointers?}

\paragraph{How do I construct Addresses?}

\paragraph{How do I concatenate Addresses?}

\paragraph{How do I create a Set of NodePtrs / NodeAddresses?}







\subsection{The IR Builder}
\label{sec:Compiler.Core.Builder}
\todo{week25}

\subsection{The Parser}
\label{sec:Compiler.Core.Parser}
\subsection{Visitors}
\label{sec:Compiler.Core.Visitors}
% do not forget visiter support for address paths
\todo{week26}
\subsection{Mappers}
\label{sec:Compiler.Core.Mappers}
\todo{week28}
\subsection{Lang-Basic and Extensions}
\label{sec:Compiler.Core.LangBasic}
\todo{week29}
\subsection{The Printer}
\label{sec:Compiler.Core.Printer}
\todo{week27}
\subsection{The Dumpers}
\label{sec:Compiler.Core.Dumpers}
\subsection{The Type Deduction}
\label{sec:Compiler.Core.TypeDeduction}
\subsection{The Semantic Checks}
\label{sec:Compiler.Core.SemanticChecks}
\todo{week31}
\subsection{Analysing Utilities}
\label{sec:Compiler.Core.Analysis}
\todo{week26}
\subsection{Node Statistics}
\label{sec:Compiler.Core.Statistics}
\subsection{Manipulation Utilities}
\label{sec:Compiler.Core.Manipulation}
\todo{week30}
\subsection{IR Data Encoding}
\label{sec:Compiler.Core.Encoding}
\subsection{Arithmetic Utilities}
\label{sec:Compiler.Core.Arithmetic}


\section{Annotations}
\label{sec:Compiler.Core.Annotations}

\section{Frontend}
\label{sec:Insieme.Frontend}

The frontend of the compiler is responsible of parsing an input program written
in a specific input language and to produce an IR program which semantically
equivalent to the input code. Because the IR is generic, many different input
programming languages can be represented by it. However a frontend must be
specific to an input language.

In the current development stage of the Insieme compiler only one frontend
exists for C-like languages based on the LLVM/Clang compiler~\cite{clang}. This
frontend supports C and C++, and above that it can deal with C language
extensions which are for example utilized for OpenCL and OpenMP. Because the
language extensions are always defined on top of a valid C/C++ code, the frontend
parse the input code in 2 major steps. In the first step, the sequential part of
the code is treated and converted into IR data structures. During this process,
information of eventual code extensions is collected and stored internally to
the frontend. In the second step, language extensions encountered in the input
code are applied to the generated IR and the final IR program is produced.  

\subsection{Overview of the Insieme Frontend [Simone]}
\label{sec:Insieme.Frontend.Overview}

The frontend's job is to convert the AST generated by the LLVM/Clang compiler
into the corresponding IR DAG. Before this conversion can take place, the AST of
an input C/C++ code has to be generated. An overview of the conversion chain
implemented in Insieme is depicted in Figure~\ref{fig:Frontend.Architecture}.
The major difference between how any C compiler works and Insieme starts here.
While a generic C compiler parses, analyzes and compiles each translation unit
of the input program separately (often for performance reasons); Insieme needs
to have the knowledge of the entire input program before the conversion can be
started. For example, in order to be constructed, a \type{CallExpr} node of
the IR needs a reference to the corresponding \type{LambdaExpr} node which
contains the body of the invoked function.  Therefore the \texttt{CallExpr} node
cannot be created before the invoked function has been converted. Because a
function body in a C program often refers to a function definition in a
different translation unit, all the content of the translation units composing
the input program needs to be collected before the IR conversion process can
start.  This part will be discussed in more details in
Section~\ref{sec:Insieme.Frontend.TranslationUnits}.

\begin{figure}[tb]
	\centering
	\includegraphics[width=\textwidth]{compiler/frontend/architecture.pdf}
	\caption{Overview of the frontend architecture}
	\label{fig:Frontend.Architecture}
\end{figure}

Once all translation units AST are in memory an analysis step on the entire
program is performed to capture eventual global and static variables in the
input code.  Indeed, because of the structural nature of the INSPIRE language,
variables are not referred by their name but instead by the DAG node
representing that variable. This makes it difficult to handle the semantics of
C/C++ \emph{global} and \emph{static} variables. In order to create a valid, and
semantically equivalent, IR program, the frontend needs to remove every global
variable from the program and accordingly replace them with plain variables
taking care of maintaining the semantics of the code. The details and
implementation issues related to the analysis phases performed for this issue
are discussed in Section~\ref{sec:Insieme.Frontend.Global}. 

Conversion of the Clang AST into an IR DAG is done using the well established
``Visitor'' design pattern~\cite{visitor-pattern}. The main idea is, for each
node of the Clang AST, to provide a transformation function which describe how
the C language entity (e.g. a variable declaration, an expression) should be
represented in the INSPIRE intermediate representation.  Management code makes
sure the generated IR nodes are automatically/correctly composed into a DAG. The
conversion of AST nodes of the C/C++ language is separated (for readability
issues) into 4 modules. 
\begin{description}
\item [\type{TypeConverter}] takes care of converting C/C++ data types (e.g. int,
array, struct) into the
corresponding IR types;
\item [\type{StmtConverter}] takes care of converting C statements (e.g. for, if,
switch) into the corresponding IR statements;
\item [\type{ExprConverter}] converts C expressions into IR expressions;
\item [\type{CxxConverter}] Converts C++ specific entities (e.g. virtual method
calls) into an IR representation.
\end{description}
These four modules are managed by the \type{ConversionManager} which is described
in Section~\ref{sec:Insieme.Frontend.Convert}
While the conversion of most of the C AST nodes is straightforward and heavily
documented in the source code, one of the challenges in the frontend is the way
recursive types and functions are generated. Handling of such problem is treated
in Section~\ref{sec:Insieme.Recursion}.

One of the major features, and efforts, of the Insieme frontend is the handling
of user pragmas. Indeed, as part of the Insieme project, an engine for pragma
matching on top of the Clang compiler was developed. This framework allows for
user pragmas to be easily defined in EBNF form. The engine takes care of
matching those pragmas and store in a separate data structure, as node
annotations, the pragma information for later
consumption~\ref{sec:Insieme.Pragmas}. A use case is the implementation of the
entire OpenMP 3.0 standard on top of our framework~\ref{sec:Insieme.OpenMP}.

During the conversion the frontend stores in the IR nodes, using annotations,
several information which can be used by the middle- and back-end to gain more
knowledge of the input program. An example are OpenMP annotations which are used
to store the information contained in the OpenMP pragmas present in the input
code. In equivalent way, annotations are used to store OpenCL attributes whose
semantics is then handled in the OpenCL compiler~\ref{sec:Insieme.OpenCL}.

\subsection{Multiple Translation Units [Simone]}
\label{sec:Insieme.Frontend.TranslationUnits}

INSPIRE represents input programs as a whole. This opposes to the way C programs
are usually written, by splitting the entire program into multiple files or
\emph{translation units}. In the trivial case when the entire program is
contained into a single source file, then the generation of the INSPIRE program
can be generated by examining that single file.

\begin{figure}[tb]
	\centering
	\includegraphics[width=\textwidth]{compiler/frontend/tran_units.pdf}
	\caption{Class Diagram of the Frontend's entry point}
	\label{fig:Frontend.Translation.Units}
\end{figure}

\subsubsection{{\tt LLVM/Clang} Compiler Wrapper}

The Insieme frontend is shaped around the {\tt LLVM/Clang} compiler which
provides the utilities to perform syntactic and semantic analysis on the input
code. Because of performance reasons, the {\tt LLVM/Clang} compiler parses
translation units separately.  An instance of the {\tt LLVM/Clang} compiler
takes care of converting a C/C++ input file into an AST which is internally
represented by an object of the class \type{clang::ASTContext} 
{\tt [\url{http://clang.llvm.org/doxygen/classclang_1_1ASTContext.html}] }. 
In order to simplify the instantiation of {\tt LLVM/Clang} compiler instances,
the Insieme frontend implements a wrapper, \type{ClangCompiler} defined in the
\file{frontend/compiler.h} header providing a simple way of retrieving a {\tt
LLVM/Clang} AST from an input file. The class uses the PIMPL design pattern to
hide implementation details as much as possible to the consumer of this class.
The code which deals with the instantiation of a Clang compiler instance and the
setup of compilation flags being forwarded to Clang is isolated in the
\file{frontend/compiler.cpp} file. In the implementation code of the
\type{ClangCompiler} class we make sure that system include paths are correctly
set both for C and C++ headers. Several other flags are forwarded from the
Insieme input flags. 

\subsubsection{Storing Translation Units}

The \type{ClangCompiler} contains the AST generated by the {\tt LLVM/CLang}
compiler. When the instance of this class is destroyed also the associated
\type{clang::ASTContext} is lost. Therefore it is important to keep alive
instances of the \type{ClangCompier} class until the conversion of the input
program to IR code is completed. Together with the AST of a translation Insieme
can also store additional data structures which refer to the translation unit
for later use. An example is the content of user pragmas within the input code.
Because {\tt LLVM/Clang} is not capable of store the information on user
pragmas, during the generation of AST we store all the pragmas into a separate
data structure \type{frontend::pragma::PragmaList} which contains the list of
pragmas in the current translation unit; where each pragma points to the AST
statement it was associated to. Once a translation unit is processed, all the
information are stored in the \type{frontend::TranslationUnit} class. 

\subsubsection{Frontend's Main Entry Point}

The task of keeping alive translation units is performed by the
\type{frontend::Program} class defined in \file{frontend/program.h}. Also this
class uses the PIMPL design pattern to hide its implementation details. The
interface of this is the main entry point of the Insieme frontend. The
constructor of the \type{Program} class accept a \type{core::NodeManager}, which
will be used during the conversion from C to IR.  The method
\decl{addTranslationUnit(const std::string\& file)} has the purpose of loading
the AST of the {\tt file} into memory. When all translation units are loaded,
the \type{convert()} function triggers the conversion of the input program into
an IR DAG. 

An example of how to manually instantiate the frontend: 
\begin{srcCode}
using namespace insieme;

using core::NodeManager;
using frontend::Program;

NodeManager mgr;
Program p(mgr);
p.addTranslationUnits( {"file1.c", "file2.c"} );
// Use the settings provided by he input line arguments 
core::ProgramPtr ir = p.convert();
\end{srcCode}

This way of initializing the frontend requires command line options, which for
example contains the list of include folders and preprocessor definitions, to be
previously set (see \ref{Command.Line.Args}). Another way of invoking the
frontend overwriting the values set via command line options is
provided by the \type{frontend::ConversionJob} \emph{facade} defined in
\file{frontend/frontend.h}.

\begin{srcCode}
using namespace insieme;

using core::NodeManager;
using frontend::ConversionJob;

NodeManager mgr;
ConversionJob job(mgr, {"file1.c", "file2.c"}, {".", "/usr/include"});
// Enable OpenMP support
job.setOption(frontend::ConversionJob::OpenMP, true);
core::ProgramPtr ir = job.execute();
\end{srcCode}

When a new translation unit is add, the parser of the {\tt LLVM/Clang} compiler
is invoked on that file and AST is generated. This action is triggered by the
constructor of a \type{TranslationUnitImpl} object which is defined in
\file{frontend/program.cpp}. The constructor of this class takes care of
registering pragma handler (see Section~\ref{sec:Insieme:Pragmas}) and starting
the parser which perform syntactic and semantic checks on the input code. If the
translation unit contains no errors, the \type{clang::ASTContext} object is
returned. 

The next operation performed on the AST associated to the translation unit is
\emph{indexing}. Indeed, because during the IR generation we need to be able to
retrieve, by name, symbols which may have been defined in a different
translation units we need to generate an index, or symbol table, which allows us
to easily find definitions given a name. Fortunately, the {\tt LLVM/Clang}
compiler provide an indexing utility \type{clang::idx::Index}.

\subsubsection{Symbol Index and Function Call Graph}
Once every translation unit is loaded, and the index is populated the last
action before the conversion starts is to locate the main entry point of the
input program. Insieme (at this development stage) can only correctly deal with
input codes having an entry point. For example, Insieme cannot be used to
compiler a library code. The reason is mostly connected with the design
of the IR which enforces restrictions on the way global and static variables are
used. We cover this aspect in detail in the next
Section~\ref{sec:Insieme.Frontend.Global}. In order to locate the entry point of
the input code we generate the whole call-graph of and then locate the entry
point. This is done using the \type{clang::CallGraph} utility provided by the
{\tt LLVM/Clang} compiler. If the input program has not entry point, the
frontend launch an exception and quite the compilation process. If the main
entry point is present, then the C/C++ to IR conversion is started. 

\subsubsection{Implications}
The way Insieme handle multiple translation units has several implication on
what Insieme ``can'' and ``cannot'' do. For example, when compiling a big
project via a {\tt Makefile} it {\bf WILL NEVER} be possible to replace the {\tt
CC} and {\tt CXX} environment variables to point to the Insieme executable and
run {\tt make}. As already stated, Insieme needs all the translation units of
the input program to be specified before the conversion to INSPIRE can be
performed. 

Two strategies can be used to compile programs with multiple translation units 
in Insieme. The former is to list all the source files composing the input
program when the Insieme compiler is launched. 

\begin{verbatim}
$> insiemec file1.c file2.c file3.c main.c 
\end{verbatim}

This solution works with small programs, however fails for large codes for
several reasons. First of all, it requires to list all the files of the input
program which depending on the complexity of the project could be located in
several paths. Additionally, real codes usually do not compile all the source
codes in the {\tt src} folder but depending on user-provided compilation flags
choose a version of a file instead of another. Often trying to compile all
source files in a project will result in a compilation error (even when a
standard C/C++ compiler is utilized). 

A more appealing strategy which can be used in such scenarios is to trick the
{\tt make} command and instead of letting him produce a binary file, let it put
together all the source code of the input program into a single file. The
output of the main will be a huge blob of source code which will contain a main
entry point and therefore can be handled by Insieme. This solution is in
practice very easy to be applied. First of all instead of running the host
compiler on every translation unit, let make run only the preprocessor. In this
way all the preprocessor macros are taken care of and the output file will be a
source code where all the headers and macros have been expanded. In order to do
this set the {\tt CC} or {\tt CXX} compiler to {\tt gcc -E}. Notice that the
object files {\tt .o} generated for each translation units will now be source
code. The last step of the makefile is to run the linker. Because the linker
expects object files, and instead we have source files we need to replace the
linker command with an utility which merges together the source code into a
single output file. This might be more complex than expected as the make file
often uses the {\tt CC} variable also as a linker. In this case the linker will
fail because GCC assumes the {\tt .o} files are object files and it will not be
able to perform any operation on those files. If you are able to locate the
linking command in the Makefile, you can replace the {\tt LD} variable with a
merging utility like the following:

\begin{lstlisting}
#! /bin/sh
""":"
exec python $0 ${1+"$@"}
"""
from optparse import OptionParser
import shutil

parser = OptionParser()
parser.add_option("-o", "--output", dest="out", help="output filename", metavar="FILE")
(options, args) = parser.parse_args()

out_file = options.out
for obj_file in args:
	shutil.copyfile( obj_file, out_file )
\end{lstlisting}

The output file of the make will be an executable which contains the entire
application source code. This singol file can be fed to Insieme and let the
magic happen.



\subsection{Global/Static Variables [Luis and Herbert]}
\label{sec:Insieme.Frontend.Global}

Global scope variables are complicated to handle during the translation because the whide scope and
the diferent translation units where the definition might be.\\
To make it a little more complicated, there are several kind of globals and they might produce
errors if not handled nicely.

\subsubsection{previous versions of INSPIRE}

Previous versions of INSPIRE made use of a so called globals elimination algorithm. The idea behind
this procedure was to eliminate the global access to variables and substitute it by an structure
which groups the storage of all those elements, this struct was allocated in main function stack and
passed by reference to all functions which needed it, directly or indirectly. 
Such structure was initialized at the begining of the program.

The main reason for not having native support for global variables in Insieme was
the fact that INSPIRE was designed to support parallel analysis and code
transformations. Indeed, having global and static variables forces the
middle-end of the compiler to deal with such concepts thereby adding more
complexity to the analysis and transformation modules.

This aproach is still valid, but such transformation should be done in the IR and not during
original translation.

\subsubsection{Regarding globals}

Using globals in parallel programs leads to syncronization problems and they my be a source of race
conditions. They should be avoided but reality is that they are widely used by every kind of
programmers. Even new programing languajes for parallel computing count with spetial global
variables (CUDA or openCL).

Related work on why it is important to avoid global variables in parallel programs can be found in
\cite{Zheng:2011:AHG:2117686.2118457}. 

\subsubsection{The Global variable collector}

Globals need to be localizad and counted across different translation units to find out their scope
and visibility.  All translation units are analyzed, for each variable declaration, it is stored. 
All variables declared inside of a function which have global storage are Statics, the rest are
globals. If no definition is found, it should remain Extern.

\subsubsection{Generating a Global Var in IR}

Globals are not definied within the IR, because IR only covers the code inside of the execution
tree. They can be just called by name, making use of a literal node, named as the global, and typed
with the right type.

\textit{lookupVariable} function in basicconvert.cpp takes care of the generation and caching of every
variable used within the program, therefore whis is the right place to build this spetial construct.

\subsubsection{Globals Initialization}

Once whole program has being translated, is time to initialize those globals which have some default
value at their's declaration. All other which have only memory allocation are not needed to be
initialized as the standar dictates. 

At the beginig of the main function, all those globals with initialization are assigned the right
value/expression. Notice that this is an assigment, so the left side must be typed
\textit{ref<type>} while the right side is \textit{type}

\begin{srcCode}
int global = 0;

int main(){
	static int a;  
	a++;
	globalmain++;
	return 0;
}

/////////////////////////////////
// turns to be something like:

let fun001 = fun() -> int<4> {
    global := 0;
	a := CreateStatic(type<int<4>>);

	gen.post.inc(AccessStatic(a));
	gen.post.inc(global);

	return 0;
};

// Inspire Program 
//  Entry Point: 
fun001

\end{srcCode}


\subsubsection{How the backend proccedes [HERBERT]}

\subsubsection{Globals and OpenMP}

All global Variables can be marked \textit{thread private} using an omp pragma. This variables IR
representation needs to be annotated so the OpenMP transformations still know about this issue.

\subsubsection{Globals Relatives}

Globals are present in different flavours:

\textbf{Extern variables}\\
	Those are variables which are declared without definition but can be used in the current
	translation unit. The global collector searches for declarations in all translation units and
	update the storage type to global if the same symbol is defined in other TU. Otherwhise will
	remain extern and it will not be declared extern by the backend so the definition needs to be
	linked by the backend compiler.\\
\textbf{Static variables}\\
	This variables have local visibility within the function where they were declared. They can
	produce aliasing problems because they can have common name with a global. Backend could find
	some trouble to diferenciate those since the name of the symbol is the same in both global and
	function static literal, if the type is also equal, there will be no chance to diferenciate
	them.
	That is why static variable names are modified with an incremental counter, so they turn unique.

	Because of the spetial initialization forced by standard, it is needed to wrap this variables in
	an spetial type which holds the flag for the unique initial initialization. 
	They need to be: created in main (to zero initialize the flag), initialized at the first
	function call (if they have initialization) and unwrapped within every read/writte access (this
	is done automaticaly by \textit{lookupVariable})

\begin{srcCode}

	int f(){
		static a= 0;
		return a++;
	}

	int main(){
		f();
	}

/////////////////////////////////
// turns to be something like:

	let type000 = struct __static_var <
		initialized:bool,
		value:int<4>
	>;

	let type001 = struct __static_var <
		initialized:bool,
		value:'a
	>;

	let fun000 = fun() -> int<4> {
		InitStatic(static_a0, 0);
		return gen.post.inc(AccessStatic(static_a0));
	};

	let fun001 = fun() -> int<4> {
		static_a0 := CreateStatic(type<int<4>>);
		fun000();
	};

\end{srcCode}

\textbf{Global Static}\\

	A global static is a variable which visibility is only the current traslation unit.

	\textbf{wow!} this is not implemented, the problem will only happen if two variables are declared static
	global in two different translation units and they share name (including namespaces if any)

	







\subsection{User Pragmas [Simone]}
\label{sec:Insieme.Frontend.Pragmas}

One of the main features of the Insieme frontend is the ability to easily define
new pragmas. For this purpose a framework has been written which allows the
definition of user pragmas similarly to EBNF form. The framework takes care of
matching pragmas in the input code against the specification given by the user.
If the pragma cannot be matched, an error is produced. Otherwise, if the pragma
is compatible with the EBNF specification, an annotation node is automatically 
generated and associated to the corresponding Clang AST to which the pragma is 
referring. 


\subsubsection{Registering Pragma Handlers}

The implementation of the pragma handling framework is located in the namespace
\decl{frontend::pragma}. The entry point of the framework is the
\type{BasicPragmaHandler<T>} class, define in \file{frontend/pragma/pragma.h},
which is the handler object being registered to the {\tt LLVM/Clang} parser to
be invoked when a pragma is encountered in the input file. In order to
facilitate the creation of such handler objects, the
\type{frontend::pragma::PragmaHandlerFactory} class is defined. 

An example of how a new pragma handler is specified follows:
\begin{srcCode}
clang::PragmaNamespace* omp = new clang::PragmaNamespace("omp");
pp.AddPragmaHandler(omp);

PragmaHandlerFactory::CreatePragmaHandler<OmpPragmaSection>(
	pp.getIdentifierInfo("section"), tok::eod, "omp"
);
\end{srcCode}

First of all {\tt LLVM/Clang} needs a \type{clang::PragmaNamespace} object to be
created which represents the base-name of the pragma. This must be the string
which follows the ``\#pragma'' keyword in the input program. In the example we
define an handler for \srcCodeInl{#pragma omp section}, therefore the namespace
is defined to be \srcCodeInl{omp}. After the namespace is created, we register
it by adding the handler to the current preprocessor ({\tt pp}) which can be
retrieved by the \type{frontend::ClangCompiler} class (note that the Clang
preprocessor shall take ownership of the provided pointer, therefore there is no
need to cleanup the memory, this will be done by Clang once the preprocessor is
destroyed). The code then generates a specific handler for the ``section''
keyword using the \type{PragmaHandlerFactory}. The registration requires the
user to provide the keyword for which this handler must be invoked, an EBNF
specification (which we will explain later), and the name of the namespace
provided as a string. Additionally a template parameter must be provided which
represent the class being instantiated to hold the informations contained on the
matched pragma (\type{OmpPragmaSection} in the example). 

For new pragmas, the user should define a class in order to process and store
the user data contained in the pragma itself. The class
\type{frontend::pragma::Pragma} defined in \file{frontend/pragma/pragma.h}
provide a base class for such purpose. A pragma is defined to store the location
of the start and end location and a reference to the Clang node to which the
pragma was attached. In C a pragma can refer to two kind of nodes, either
declarations (e.g. \type{clang::FuncDecl}, \type{clang::VarDecl}) or a generic
statement. The methods \type{isDecl()} or \type{isStmt()} of the
\type{frontend::pragma::Pragma} class shall be utilized to test the type of the
target node. The methods \type{getDecl()} or \type{getStmt()} can be used to
retrieve a pointer to the target node. 

\subsubsection{Overview of Pragma Matching}

The constructor of the class provided to the \type{CreatePragmaHandler} function
is invoked automatically by the framework once a pragma is being matched. The
matching process is split into two phases.

\begin{description}
\item [Phase 1]: In Phase 1 the framework tries to match the content of the
pragma against the EBNF specification provided by the user. This is implemented
using a standard backtracking engine which consumes the input stream until the
``end of directive'', {\tt clang::tok::eod}, token is encountered. If the
matching cannot be performed, an error is produced and the pragma is discarded.

If instead the pragma is correctly matched, an instance of the user provided
object type is generated and stored in a list of ``pending'' (or unmatched)
pragmas.

\item [Phase 2]: The Phase 2 takes care of attaching pragmas to the
corresponding statements. Because {\tt LLVM/Clang} processes the pragma before
looking at the target statement (and therefore an AST node is not available by
the time the pragma is processed), the association must be performed lazily.
Because the lack of any context information when a pragma handler is being
invoked, the matching is performed solely on the basis of source code
locations. 
\end{description}


\subsubsection{Pragma Specification}

A pragma specification is provided to the system as an expression built using
C++ operators in a way which resembles the EBNF form. The specification
expression composes a tree which is implemented following the composite design
pattern for which the \type{frontend::pragma::node} class (defined in
\file{frontend/pragma/matcher.h} is the abstract base. The leaves of the
generated tree are lexer tokens. A pragma specification can be built using the
following 4 operators:

\begin{description}
\item [{\tt t1 >> t2}:] Binary operator which represents the concept of
\emph{``concatenation''}, it matches the input stream the next two tokens are
respectively {\tt t1} followed by {\tt t2};

\item [{\tt t1 | t2}:] Binary operator which represents the concept of
\emph{``choice''}, it matches the input stream if the next token is either token
{\tt t1} or token {\tt t2};

\item [{\tt !t}:] Unary operator which represents the concept of
\emph{``option''}, it matches the input stream if the token {\tt t} is present 0
or 1 times;

\item [{\tt *t}:] Unary operator which represents the concept of
\emph{``repetition''}, it matches the input stream if the token {\tt t} is
present 0 or N times;
\end{description}

In each case a token {\tt t} can be either a lexer token (leaf node of the
expression) or an expression. Using these operators is possible to define, for
example, the {\tt for} clause of an OpenMP for (the full code can be found in
\file{frontend/omp/omp\_pragma.cpp}.

\begin{srcCode}
auto kind =  
	Tok<clang::tok::kw_static>() | kwd("dynamic") | kwd("guided") | 
	kwd("auto") | kwd("runtime");

auto op = tok::plus | tok::minus | tok::star | tok::amp |
		  tok::pipe | tok::caret | tok::ampamp | tok::pipepipe;

auto var_list = var >> *(comma >> var);

auto reduction_clause = kwd("reduction") >> 
	tok::l_paren >> op >> tok::colon >> var_list >> tok::r_paren;

auto for_clause =	
	    reduction_clause
	|	(kwd("schedule") >> tok::l_paren >> kind >> 
		!( tok::comma >> expr ) >> tok::r_paren)
	|	(kwd("collapse") >> tok::l_paren >> expr >> tok::r_paren)
	|   kwd("ordered") | kwd("nowait") 
	;
\end{srcCode}

As already stated, the leaf nodes of the expression are lexer tokens which are
imported from {\tt LLVM/Clang} token definition (see in the
\file{clang/Basic/TokenKinds.def}) and made available under the
\decl{frontend::pragma::tok} namespace. Additionally to the preprocessor tokens
of clang we define new leaf nodes for the purpose of simplifying the
specification of new pragmas. 

\begin{description}
\item [kwd( str\_lit ):] the matcher expect to encounter an identifier which is
exactly the literal provided as argument. Note that because we use the C lexer,
keywords which are recognized to be reserved words in the C/C++ language cannot
be matched in this way. In such cases the name of the Clang token must be used,
for example {\tt clang::tok::kw<static>} represent the ``static'' identifier. 

\item [expr:] This placeholder matches any C/C++ expression. Indeed, usually
pragmas may contain expressions. One limitation of the {\tt LLVM/Clang} pragma
matching mechanism is that the C parser is not made available to pragma
handlers. However this is more of a Clang design limitation rather then
capabilities. In order to overcome this problem Insieme works on a patched
version of the {\tt LLVM/Clang} compiler.  The patch makes the engine for pragma
matching of insieme friend (in a C++ way) with the \type{clang::Parser} object.
This enable us to be able to access to the Clang parser also when pragma
handlers are invoked and therefore be able to parse complex C expression without
reinventing the wheel. 

\item [numeric\_constant:] Another useful placeholder which specifies that the
token to match must be any valid numeric constant. 

\item [var:] Makes sure that the matched token is a valid variable. This not
only assures that the token is a valid C identifier, but also that the variable
has been declared. Use of undeclared variable will be captures as an error by
the preprocessor. 
\end{description}

Matching the structure of the pragma is just a part of the whole story. Indeed,
the user may be interested not only to know that a statement has associated a
particular type of pragma, but, most likely, it may be interested to its content. 
Usually, the information contained in the pragma
are mostly syntactic sugar which, once the pragma has been matched, loose any
function. Because the framework cannot decide by itself what is
interesting and not for the user to be stored, we define two additional
operators which allows the user to specify what should be extracted from a
particular pragma once is matched. 

\begin{description}
\item [{\tt["key"]}:] At any point of the pragma specification the {\tt []}
operator can be used to force the framework to store all the tokens matched by
the node to which the operator is applied. Informations are store into a
multimap where the value of the key is the string value provided as argument of
the {\tt []} operator. 
For example, \srcCodeInl{(var >> *(comma >> var))["VARS"]} stores all matched
tokens into a map having {\tt "VARS"} as a key and the list of matched tokens as
value. If the following input is encountered: \srcCodeInl{a, b, c} the resulting
map will be of the form: {\tt ("VARS" -> \{ a, b, c, "," \})}. 

\item [\~{}:] As seen before, sometimes we want to be able to {\em exclude}
specific type of tokens to be mapped to the resulting result. This is the
purpose of the \~{} operator which forces any token mapped by the addressed
expression to be removed from the mapping. Therefore, by changing our
specification for variable lists to: \srcCodeInl{(var >> *(~comma >>
var))["VARS"]}, the resulting multimap will be the following:  {\tt ("VARS" ->
\{a, b, c\})}.

\end{description}

The result of the matcher is therefore an object of type
\type{frontend::pragma::MatchMap}, which is defined in
\file{frontend/pragma/matcher.h}. Each key is matched to a list of objects which
can be either a pointer to a string (used when the pragma matches string
literals for example) or a pointer to a generic \type{clang::Stmt*}. This is the
case when the matched token is a C expression or for example a variable
identifier. It is worth noting that by default keywords nodes are inserted into
the matching map without the need for the user to explicitly specify the
mapping. A \srcCodeInl{kwd("auto")} for example will create an entry in the map
whose key is {\tt "auto"} and the value is an empty list. A recurring use case
is represented in the following code snippet:

\begin{srcCode}
auto var_list = var >> *(~comma >> var);

auto private_clause = 
	kwd("private") >> tok::l_paren >> var_list["private"] >> tok::r_paren;

auto for = 
	kwd("for") >> !private_clause;
\end{srcCode}

Where the {\tt "private"} keyword is utilized to capture the list of variables
associated to the clause. Given the following pragma \srcCodeInl{#pragma omp for
private(a,b)} the generated matching map will be the follow: {\tt "private" -> \{
a, b \}; "for" -> \{ \} }

The generated map is forwarded to the object constructor registered through the
{\tt PragmaHandlerFactory}. The pragma object is constructed iff the matcher is
able to completely match the user specification against the input code. In the
contrary case, an error is automatically produced which highlight the location
on which the backtracking algorithm stopped because couldn't find any rule to
proceed with the matching. In those cases, the algorithm keeps a list of
options at that particular location which is printed out in the error message.

The user object is responsible to analyze the matching map returned by the
pragma matcher and do the necessary operations. An explanatory example of how
OpenMP pragmas are processed and stored is in the
\file{frontend/pragma/omp/omp\_pragma.cpp} file. 

\subsubsection{AST Node Mapper}

The second phase of the pragma framework takes care of associating, or mapping,
generated pragma objects to the AST nodes to which pragmas refer. The framework
does this operation in a way which is transparent to the user. The code which
takes care of this aspect is in the \file{frontend/sema.h} and
\file{frontend/sema.cpp} files. 

Two solutions are possible to connect pragmas with statments; one solution would
be to traverse the entire AST after it has been generated.  However this
solution requires an expensive traverse of the input program AST and this could
be inefficient for big codes, especially when very few pragmas are present in
the input program. 

The way the {\tt LLVM/Clang} compiler builds the AST is by invoking ``actions''
provided by the \type{clang::Sema} class which reduces the tokens currently
available on the parser stack and generates the corresponding AST node.
This is the best location for implementing our matching algorithm for pragmas.
Indeed we can keep the list of processed pragmas in a list and every time a new
statement is being created by the \type{clang::Sema} class we check, based on
the location, whether any of the pending pragmas refer to the newly generated
statement. With this solution the overhead produced by the matching is minimal. 

Fortunately, when the first version of insieme was developed, Clang interfaces
allowed for the class \type{clang::Sema} to be extended. As a matter of fact all
the action methods were virtual allowing for extending its behaviour.
Starting with clang 2.9 this was changed and the developers removed the
possibility to customize the actions of the \type{Sema} class. In order to
overcome this limitation imposed by the Clang developer we patch the clang code
making virtual the functions of the Sema class for which we need to redefine the
behaviour. 

As stated before, the matching algorithm works with the only context information
available at this stage, i.e. locations. The check for pragma matching is
performed for few selected node types, i.e. {\tt ActOnCompoundStmt, ActOnIfStmt,
ActOnForStmt, ActOnStartOfFunctionDef, ActOnFinishFunctionBody, ActOnDeclarator, 
ActOnTagFinishDefinition}. The algorithm keeps a list of pending pragmas ordered
by their locations. Once one of the previous statements is reduced, we check
which pragmas are within the range of the statement. If none, we check whether
any of the pending pragmas are located right before the statement. In that case
we associated the pragma to that statement and we remove the pragma from the
list of pending pragmas. If pragmas are located within the range of the
statement we iterate through the children and match according to the positions.

\subsubsection{Detached Pragmas} 
The tricky aspect of the entire algorithm is how we deal with pragmas which are
not meant to be attached to a statement. An example is the OpenMP \srcCodeInl{#pragma
omp barrier}. Indeed this pragma is not meant to be associated to a statement,
and the following code is a valid OpenMP input code:

\begin{srcCode}
{
	...
	#pragma omp barrier
}
\end{srcCode}

Because the {\tt LLVM/Clang} compiler doesn't represent pragmas in the AST we
need a way to easily map the location of a pragma to a node in the AST. This is
way we map pragmas to statements, so that when a statement of the Clang AST is
visited we can easily check whether there is a pragma associated to that
statement. However, when a pragma doesn't not refer to a statement we need to
create an empty statement in the AST (a {\tt no-op}, {\tt ;}) and associate the pragma
to that statement. 

\subsubsection{Traverse (and Filter) Pragmas}
\todo{soon}


\subsection{Conversion Manager [Simone]}
\label{sec:Insieme.Frontend.Convert}

\begin{figure}[tb]
	\centering
	\includegraphics[width=0.8\textwidth]{compiler/frontend/seq_frontend.pdf}
	\caption{Dataflow of the Insieme Frontend}
	\label{fig:Frontend.Seq}
\end{figure}

In Figure~\ref{fig:Frontend.Seq}, the flow of execution of the frontend is
depicted. We already explained two component which are executed before the
actual conversion into IR is performed. The task of converting the {\tt
LLVM/Clang} AST is done by the \type{frontend::Conversion\-Manager}. In order
to reduce the complexity of the conversion procedure, we split the code into 4
"converters" taking care of specific aspects of the C language:

\begin{description}

\item [BasicConverter:] Defined in \file{frontend/basic\_convert.cpp}. It
contains utility functions which are used across all the converters.

\item [TypeConvert:]  Defined in \file{frontend/type\_convert.cpp}. It takes care
of the conversion of C/C++ types into IR types. 

\item [StmtConverter:] Defined in \file{frontend/stmt\_convert.cpp}. It takes care
of the conversion of C/C++ statements into IR statements. 

\item[ExprConverter:] Defined in \file{frontend/expr\_convert.cpp}. It takes care
of the conversion of C/C++ expressions into IR statements. 

\end{description}

To make the code readable, the implementation of the 4 converters are spread
across 4 translation units. Also, because of optimization issues related to the
amount of symbols being exported by the frontend {\tt .so} library, the
definition of those converters is completely hidden and not accessible outside
the frontend.  Because the \type{frontend::Con\-ver\-sion\-Manager} provides a facade
for invoking the conversion facility, so there is no need for external access of
the single converters. Communication among converters is obtained through the
manager class, this means that each converter can utilize functionalities of
another converter (or himself) through the \type{frontend::ConversionManager}
class whose interface is defined in \file{frontend/convert.h}.
Indeed the manager class provides 3 fundamental methods: 

\begin{description}
\item [{\tt TypePtr convertType(const clang::Type*)}:] Converts an {\tt
LLVM/Clang} type node into an INSIPRE type.

\item [{\tt ExpressionPtr convertExpr(const clang::Expr*)}:] Converts an
{\tt LLVM/Clang} expression node into an INSPIRE expression. 

\item [{\tt StatementPtr convertStmt(const clang::Stmt*)}:] Converts an
{\tt LLVM/Clang} statement node into an INSPIRE statement.

\end{description}

Beside taking care of dispatching the conversion task to the appropriate
converter, the manager also performs some optimizations at this stage in order
to speedup the conversion process. For example, by introducing caching we avoid
to convert symbols which have already been converted. Caching is not utilized
for all types of IR nodes as the probability of converting two identical
statements is very low. However, caching is quite successful for types nodes as
the type reference to type node ration is high. Future PhDs might be interested
in speeding up the frontend time and introducing smarter way of caching nodes
could be a simple and high effective way of doing it (low hanging fruit).

Finally, the manager also retains a context which stores information necessary
through the conversion process of the IR. In the context we store those
information which needs to be visible (or alive) during the conversion of the
entire program. For example, the mapping between a Clang variable declaration
and an IR variable makes sure the same C variable is being replaced by the same IR
entity consistently for all the encountered references to it. 

\subsubsection{Type Converter}
{\tt LLVM/Clang} supports a visitor interface for traversing the AST. For this
reason the class \type{frontend::TypeConverter} inherits from
\type{clang::TypeVisitor}. The visitor defines several methods which define how
the C AST node should be transformed into IR. Each method of the visitor returns
the generated node so that the IR representation of complex node can be built
upon composition of the type node returned by successive calls to the visitor
itself. 

One peculiarity of the \type{frontend::TypeConverter} is the way we deal with
pointers type. Indeed, the IR type system doesn't support the C semantics of
pointers which not necessary refers to the element directly pointed in memory.
For example, in C a pointer can be used to refer to an array of elements or a
singular scalar variables. Ideally the
following conversion semantics shall be used for pointer:

\begin{table}
\begin{centering}
	\begin{tabular}{l|c}
		\textbf{C Type} & \textbf{IR Type} \\
		\hline \hline
		\constant{type* (R-Value)}          & \insCodeInl{ref<'type>} \\
		\constant{type* (L-Value)} 		   & \insCodeInl{ref<ref<'type>} \\
		\hline 
	\end{tabular}
	\caption{Sound type conversions for C pointers}
	\label{tab:Compiler.Frontend.ml.GenNNoutput}
\end{centering}
\end{table}

However, because we must take into account situation for which the pointer is
used to refer to an array, and the user want to access a memory location which
is not at displacement 0, then a different encoding is necessary in order to
guarantee sound semantics check of the generated IR program. For these reason
the following encoding is utilized: 

\begin{table}
\begin{centering}
	\begin{tabular}{l|c}
		\textbf{C Type} & \textbf{IR Type} \\
		\hline \hline
		\constant{type* (R-Value)}          & \insCodeInl{ref<array<'type,1>>} \\
		\constant{type* (L-Value)} 		   & \insCodeInl{ref<ref<array<'type,1>>>} \\
		\hline 
	\end{tabular}
	\caption{Implemented type conversions for C pointers}
	\label{tab:Compiler.Frontend.ml.GenNNoutput}
\end{centering}
\end{table}

\note{Proposal for 'array-erasure' procedure useful to cleanup
generated IR.}
This however introduces several levels of ugliness to the IR code being
generated by the frontend. One way to really deal with this problem is to apply
a two phase approach which shall be implemented in Insieme. Because we don't
want to perform any advanced analysis on the input code, we let the frontend
produce an IR code which contains impurities (for example representing pointers
using IR arrays as it is now). Right after the frontend completes the
conversion, we may apply an analysis (on the generated IR) which data mine the
type of usage of pointers. If for example a pointer is always used to access the
element at displacement 0 then we can safely replace the IR type to be a
\insCodeInl{ref<'type>}. Those situations where the pointer is being used to
access elements with an offset not equal zero, then the array type should be
maintained. In those cases, DEF-USE \todo{ref DEF-USE} analysis may be used to
determine the declaration of the array being addressed and if possible replace
the \insCodeInl{ref<array<'type>>} with an actual
\insCodeInl{ref<vector<'type,N>>}. All of these are options which may be
implemented right after the first phase of the frontend has been processed. It
may improve both the readability of the generated IR code and possibility for
optimizations. 

Most type conversion is straightforward, the only exception are {\tt struct}
types which may cause a problem due to the possibility to have cyclic
references. As discussed in the overview, an IR node cannot be build if the
children are not already available, therefore trying to building a node
referring to himself is not possible. The same type of problem happens for
recursive function calls and this is handled in the IR using special recursive
types and lambda definitions which will be covered
in~\ref{sec:Insieme.Frontend.Recursion}. 

\subsubsection{Statement Converter}

Conversion of statements is done in a similar way types are converted. Each
statement generates an equivalent IR statement. However it can happen that one
\type{clang::Stmt} node can generate multiple INSPIRE statements. This is the
case of declaration statements for example where one declaration statement can
declare several variables. In INSPIRE this is not allowed as, for simplicity, a
declaration statement only declare a single variable. Therefore, differently
from the type conversion, the output of a visit method of the
\type{frontend::StmtVisitor} is of type \type{frontend::StmtWrapper} which is-a
vector of statements. It is worth noting that we couldn't do any different since
by wrapping multiple statements into a compound statements for example would
have yield to different program semantics (e.g. declared variables must be
visible by following statements in the same scope). In order to simplify the
handling of the return value of the visitor methods we defined the {\tt
tryToAggregateStmts()} function which takes care of wrapping (when necessary)
the returned list of statements into one statement (by embedding them into a
compound statement). 

Most conversion of C statements are straightforward, however some of them
require some special kind of analysis in order for the conversion to be
performed. An example are loop statements. In C loop statements have a much more
complex definition than the INSPIRE for loop. For example, in C is possible to
use a for statement to implement the semantics of a while loop (using an empty
initialization statement and a boolean exit condition). Or otherwise, multiple
induction variables can be defined for a for-statement and this is also not
allowed in INSPIRE where a loop statement only contains one induction variable.
In order to determine whether a C loop statement can be represented using the
INSPIRE for statement we perform an analysis. This is implemented in the
\file{frontend/include/insieme/frontend/analysis/loop\_analyzer.h}. The
\type{frontend::analysis::LoopAnalyzer} class tries to retrieve three important
piece of information from a given for statement. 

\begin{description}
\item [Induction Variable:] The first thing is to determine the induction
variable of the for loop, this is done by extracting the set of  variables
defined in the initialization section of the for statement and intersect it with
the variable being used in the loop condition and increment operation. If the
result of the intersection is one variable, then this variable is selected to be
the induction variable for this loop. In the contrary case, an exception is
thrown in the constructor of the \type{LoopAnalyzer} class containing a message
which explains the reason why this statement cannot be represented as an IR for
statement. 

\item[Exit Condition:] Because in the IR the specification for a for loop is
based on the upper bound of the induction variable, the condition expression of a
C for statement needs to be manipulated in order to isolate the upper bound.
This is what the result of the {\tt LoopAnalyzer::getCondExpr()} method is. 

\item[Increment Expression:] The last piece of information is the value of the
increment step which is also extracted from the C definition of the for
statement. 

\end{description}

If the analyzer managers to extract without any ambiguities those pieces of
information, then the actual conversion of the for statement begins. Otherwise
an exception will be launched, of type
\type{frontend::analysis::LoopNormalizationError}, and the frontend will create
a while statement which is semantically equivalent to the for statement.  

Another big difference between C for statements and the INSPIRE for statement is
the fact the INSPIRE for always declare the induction variable and it cannot use
an existing variable for that. In such cases the frontend takes care of
introducing a new induction variable, initialize its value and at the exit point
of the for statement make sure that correct value is assigned to the original
induction variable. Details on the implementation are available in the
source code. 

Another aspect where the INSPIRE code differs from C semantics is the switch
statement. Indeed while in C the switch statement has a fall-through semantics
(which means the switch will execute the code contained in all cases following
the entry point case, unless {\tt break} is utilized), in INSPIRE there is no
such a thing. Therefore each case contains statements which are executed when
that branch is taken. This requires to copy several time the list of statement
in the case the original switch statement was written using the fall-through
semantics. 

\subsubsection{Expression Converter}

\todo{BORING stuff.}

\subsubsection{Basic Converter}

\todo{LESS BORING stuff.}

\subsubsection{Reference Wrapping}

In INSPIRE we enforces that argument of a function cannot be written unless they
are refs. A general scenario in C is a function accepting a parameter of type
{\tt int}. Later on the programmer can assign a new value to the parameter, even
though the newly set parameter value is lost (because the arguments are passed
by-value) the programmer can use this memory location instead of allocating a
new variable on the function stack. When converted in IR, a function having an
integer argument will have the \insCodeInl{( int<4> ) -> unit} type and
eventual assignment to that variable are not possible since the {\tt ref.assign}
operator only accept ref types as LHS expression. 

When such thing happens, the frontend creates a new temporary variable which is
initialized with the value of the input parameter the variable is going to
replace. However, the procedure is more complex than it would appear because of
the following design decisions. Firstly we don't want to create the temporary
variable if it  is not necessary (which means if there is no assignment to it);
secondly we don't want to pay for pre processing costs by looking for such
assignment operator before the conversion is started. 

The solution implemented in the frontend is fully functional and it introduces
zero overhead for what concerns the execution time. We start the conversion of
the code as normal, if there exist, at some point in the input code an
assignment operator to an input argument, $a$, of the function we create a new
variable, $wa$, which will be used as reference wrapper. We register the mapping
from the input argument to the newly generated IR variable in a map which is
stored within the context object \type{wrapRefMap}. Since this point, every use
of the variable $a$ will be replaced by an instance of the variable $wa$. It is
worth noting that in the code already translated into IR before the assignment
statement was encountered there might still be uses of the variable $a$ (but not
assignments to it). To fix any pending references to the old function parameter
we wait until we converted the entire body of the function and then apply some
transformation on the generated IR code. First of all we introduce a new
declaration statement at the beginning of the function body containing the
definition of the variable $wa$. The variable will be initialized with the value
of $a$. Successively we replace in the body of the function all occurrences of
$a$ with the dereferencing, \insCodeInl{ref.deref}, of variable $wa$. This is
perfectly sound as we have the guarantee that pending usages of $wa$ must be
access to the actual value.


\subsubsection{Handling of Pragma Information during Conversion}

\todo{BORING stuff.}





\subsection{Recursive Types and Functions [Simone]}
\label{sec:Insieme.Frontend.Recursion}

One of the most difficult task of the frontend is the conversion of recursive
entities like function calls and types. As already explained in the overview
section, in C those entities are addressed by their name and therefore once a
symbol has been defined, it is possible to refer to it. In INSPIRE this is not
allowed as entities are referred by their structure, and because at the moment
of creation, all the elements forming an entity must be available, creation of
recursive types is impossible. 

To solve this issues a special type and lambda construct has been introduced in
INSPIRE to allow recursive structure to be represented. The main idea behind it
is that the recursive part of a function, or a type is identified by a name and
when recursion is encountered, the name is used to refer to it. The way we deal
with recursive data types and function is the same. The code implementing this
conversion is in the \decl{convertFunctionDecl()} and \decl{visitTagType()}
functions respectively defined in \file{frontend/expr\_convert.cpp} and
\file{frontend/type\_convert.cpp}. The process is divided into the following
main steps:

\begin{description}

\item [Add Type/Function to Graph:] For every struct/class type or function
which gets converted, we add the entity to respectively a type or call graph. If
the type or function is already in the graph then returned the cached IR node
previously generated.

\item[Compute strong connected components:] If there was a change in the
type/call graph, then we run the algorithm to compute strong connected components
on the new type/call graph. In order to minimize development type, we used the
Boost.Graph library which provides a highly optimized library for representing
graphs and additionally implements several algorithms are already implemented on
it. In order to simplify the use of the Boost library we developed a small
wrapper class \type{frontend::utils::De\-pen\-den\-cyGrap} defined in
\file{frontend/utils/dep\_graph.h}. The class is generic so it can be used for
solving connected components for both functions and types. 

\item[Resolve sub-components:] The strongly connected components algorithm
returns a set of components, or cycles, detected in the given graph. We then
identify the ID of the component to which the current type/function belongs and
proceed with the conversion of all the component with a smaller ID. 

\item[Resolve current function:] Once we are sure all sub-cycles are converted
into INSPIRE lambdas/types, we proceed with the resolution of the current
function/type. Before invoking the visitor on the body of the function however
we have to change the context of the \type{frontend::ConvertionManager} to make
it aware that we are converting a recursive function, (i.e. {\tt isRecSubFunc}
for functions and {\tt isRecSubType} for types). This is important because
if in the function body we encounter a recursive call, then we should avoid to
recur again. This is done by using a boolean variable which indicates the
converter that we are resolving something which has been detected to be
recursive. 
\begin{itemize}

	\item Before recurring, and solving the function body or the stucture type
		field types, we define a set of variables (one for each entity in the
		current component) which represent a placeholder for that particular
		entity. For functions the map which associates variable names to the
		corresponding function declaration is called {\tt varDeclMap}. While for
		types the map is called {\tt recVarMap}.

	\item We then recur over the body. When a new call expression, or a struct
		type is encountred we determine whether we are already resolving a
		recursive function/type, if yes we determine whether the current
		entity belongs to the connected component. In order to optimize the
		execution time, instead of re-computing the connected components, we
		lookup the map containing the variables being associated to the elements
		of the current component. Otherwise we continue with the normal
		execution. 

	\item When the body has been converted we reset the context variable to the
		normal execution and we store in a function/type cache the converted
		IR expression to be associated to the corresponding Clang type/function.

	\item Because the definition of a recursive function or type include
		definitions also for other elements in the component, at this point we
		store in the cache the corresponding INSPIRE node who represent the
		other elements in the connected component. 

\end{itemize}

\end{description}

The algorithm might be complex but it has been tested with multiple scenario and
it has been proved to be sound. The concern of the future developer of insieme
should be extendibility to new entities. For example the current implementation
works for regular call expressions. However if CXX support needs to be ported
into insieme, the call graph needs to be extended to take into account also
constructurs, destructors, method invocations and operator overalods. This can
be done using a type trait trick. The \type{frontend::utils::DependencyGraph}
provides a \decl{Handle} method which should be extendend to handle specific
Clang nodes. This methos tells the dependency graph builder how explore a new
element added to the graph. For example if a new call expression is added, we
should add to the dependency graph eventual methods invoked within the body of
the function. 



\subsection{CXX Extensions}
\label{sec:Insieme.Frontend.CXX}
\todo{Ioannis/Bernard}

\input{frontend/10_ocl}

\subsection{OpenMP Frontend}
\label{sec:Insieme.OpenMP}



\subsection{Known Issues}

There are several known bugs/missing feature within the compiler frontend which
either can be fixed (by someone willing to spend time on them) or cannot be dealt 
with because of the IR. 

\subsubsection{Chain of Assignments}

In C, assignments can be chained in the following way:
\begin{srcCode}
a=b=c=10;
\end{srcCode}
This is currently not supported in the frontend since the code will be
erroneously converted into a composition of {\tt ref.assign} calls. Since the
return type of those functions is \type{unit}, instead of \type{a'} like in C,
this will result in a semantic error. 

The error can be easily solved by splitting the assignment into multiple
initialization statements. 

\subsubsection{Assignments within Expressions}

There is an overall problem with assignments within expressions. While the C
grammar allows assignment operations to be used as normal expressions where
their return value is the same type of the assigned variable, in IR this is not
possible since the return value of an assignment is {\tt unit}. 

\begin{srcCode}
int c = a=10 + (b=3);
\end{srcCode}

As for the previous issue with chained assignments, conversion will produce a
semantically incorrect program; fortunately the semantic checks are able to
capture such cases. 

This is a bug of the frontend which can be easily fixed by splitting the
expression into many statements. However, in order to do so we must be careful
of eventual short-circuiting within the expressions (especially for boolean
expressions) which may result in a lazy update of a variable which therefore
cannot be assigned a priori. 

A workaround is to change the input code manually so that assignments are
eliminated from expressions.

\subsubsection{Call though function pointer to function using global variables}

This is an issue which is preventing many codes from being converted into
INSPIRE. A minimal test case is the following:

\begin{srcCode}
int f() {
	static int a=0;
	return ++a;
}

int main( ) {
	int (*fptr)() = f;
	f();
}
\end{srcCode}

The function \srcCodeInl{f} is invoked through a function pointer. Since the
global variable collection done by the frontend follows the static call graph of the
input program, and the binding of \srcCodeInl{f} here is dynamic; the function will not
be visited and variable {\tt a} will not be inserted in the global struct.
Consequently the signature of the function \srcCodeInl{f} is not changed (no additional
argument is introduced to pass the global struct), and at the call site no
parameter is provided. 

In order to solve this problem a more elaborated analysis is required for global
variable collection which is not limited to the static call graph of the
program. This requires extensive analysis and changes throughout the input code
since the signature of the relative function pointer needs to be changed and
also any other function which is invoked through that pointer needs to have the
signature updated (even though no global variable is accessed within the
function body). This kind of support solves the problem when functions pointer
are used within functions for which we have a definition. If the function
accepting the function pointer is an external function then the program cannot
be converted into INSPIRE. 

\ldots


%\section{The Frontend}
%\subsection{The Frontend's Architecture}
%\subsection{The C Converter}
%\subsection{The Pragma Parser}
%\subsection{The OpenMP Frontend}
%\subsection{The OpenCL Frontend}
%\subsection{The MPI Frontend ?}
%\subsection{The C++ Frontend}
%\subsection{Customizing the Frontend}

\section{The Backend}
\subsection{The Backend's Architecture}
\subsection{The Sequential Backend}
\subsection{The Runtime Backend}
\subsection{The OpenCL Kernel Backend}
\subsection{The OpenCL Host Backend}
\subsection{Customizing the Backend}

\section{The Simple Backend}

\section{The Analysis Module}
\subsection{SCoP Analyses}
\subsection{Features}

\section{The Transformation Module}
\subsection{The Transformation Framework}
\subsection{Of Patterns and Generators}
\label{sec:Compiler.Transform.Pattern}
\subsection{The Polyhedral Transformations}
\subsection{The Rule-based Transformations}

\section{The Machine Learning Module}

\section{The Driver}
\subsection{The Measurement and Instrumentation Utilities}
\subsubsection{Remote Execution}
\subsection{The Integration Test Loader}

\section{The XML Dump}
\section{The Playground}
