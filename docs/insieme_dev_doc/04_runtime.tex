\chapter{The Insieme Runtime} \label{cap:runtime}
\index{Runtime}

\section{Overview}
\subsection{Architecture}
\subsection{Coding Standards}
In addition to the compiler coding standards \ref{sec:compiler:codingstandards} (as far as they are applicable to C), the runtime adds several other conventions that should be respected. These standards were created and ratified using the Insieme Runtime autocratic process.

\begin{enumerate}

\item \textbf{The insieme runtime is written in C99}. It should compile without warnings on any supported system, even when full warnings are enabled.

\item \textbf{Naming conventions}
\begin{enumerate}

\item \textbf{General}: Types and functions are named using the lowercase\_and\_underscores convention.

\item \textbf{Prefix}: All externally visible symbols should be prefixed with \srcCodeInl{irt\_}. (Except globals, see below)

\item \textbf{Abbreviation}: Use full names in types and abbreviations in methods operationg on them.\\
Example: \srcCodeInl{irt\_errcode irt\_wi\_delete(irt\_work\_item* wi);}

\item \textbf{Output Parameters}: Output parameters should be prefixed with \srcCodeInl{out_}. Parameters used for input and output should be prefixed with \srcCodeInl{inout_}. \\
Example: \srcCodeInl{irt\_errcode irt\_wi\_create(irt\_work\_item** out\_wi);}

\item \textbf{Typedefs}: Typedefs are good as long as they convey semantic information. However, \textbf{never} typedef a pointer. Whether some variable is a pointer or a data item should always be immediately obvious.
All IRT structures/unions should follow this scheme w.r.t. typedef:
\begin{srcCode}
typedef struct _irt_work_item {
	//...
} irt_work_item;
\end{srcCode}

\item \textbf{Globals}: Globals should always be prefixed with \srcCodeInl{irt\_g\_} and used sparingly.
\end{enumerate}

\item \textbf{Integer Types}:
Use the integer types defined in inttypes.h whenever a specific precision is
required, and \textbf{always} in basic IRT data structures.
E.g. use \srcCodeInl{int32} instead of int and \srcCodeInl{uint32} instead of unsigned.

\item \textbf{Error Handling}:
Error handling is performed using a combination of thread local error data and
signals. 

\item \textbf{Const}:
Use \srcCodeInl{const} whenever possible.

\item \textbf{Basic Data Structure Design}:
When designing basic data structures used throughout the IR, follow these
guidelines:
\begin{itemize}
\item The data structure should have a fixed, minimal size.
\item If required, the data structure should be easy to serialize and distribute
  over a network.
\item Keep the number of indirect memory accesses and redirections required to
  perform frequent operations to a minimum.
\end{itemize}
\end{enumerate}


\subsection{Source Code Organization}
\subsection{Options}
\subsubsection{Compile-time}
\subsubsection{Runtime}

\section{Utilities}
\subsection{Affinity}
\subsection{Task Switching}
\subsection{Error Handling}
\section{Workers}
\subsection{Scheduling}
\section{Work Items}
\section{Work Sharing}
\subsection{Loop Scheduling}
\section{Data Items}
\section{Event System}
\section{Instrumentation}
\section{Customizing the Runtime}

\section{Following an Execution}
List start-up, run and shut-down steps to obtain an idea where to include
potentially required mods \ldots
