\chapter{Introduction} \label{cap:introduction}

Thank you for your interest in using the Insieme \cite{insieme} project for your
research. This document will help you getting familiar with the individual
components, tools and utilities designed and developed to support your research
in the area of analysing, transforming, tuning, executing and monitoring
parallel, close-to-real-world codes.


The Insieme project aims to provide an easy to use, powerful foundation for
research on high-level code analyses, transformations, optimizations and (auto-)
tuning with a particular focus on (heterogeneous) parallel applications.
It provides a foundation for research in the area of static and dynamic
program optimization as well as hybrid solutions involving static and
dynamic elements. The static part is realized by the \textit{Insieme Compiler}
toolkit while the dynamic contributions are incorporated by a
\textit{Insieme Runtime} system.   

 
In the compiler kit, the features supported by the various composable components
range from frontends supporting multiple, potentially mixed input languages and
customizable pragmas, over an uniform internal program representation, an
extensible set of analysis and manipulation utilities to customizable backends.
Additionally, the runtime system allows to control various aspects of the
parallel execution of applications, including the degree of parallelism, the
affinity of threads, work-load distribution and the memory management. Further,
the runtime provides online monitoring capabilities to support decision making
processes while steering the execution of program code. Finally, hybrid
utilities like the instrumentation and execution of program code for obtaining
dynamic features (e.g. the execution time, energy consumption\footnote{Planned,
yet so far not implemented} or hardware counters) are supported and may be
exploited by an iterative compilation schema.

\section{The Purpose of this Document}
The purpose of this document is to establish a uniform documentation and
reference for the various contributions in the context of the Insieme project.
This document is maintained together with the source code repository and
continuously extended whenever new components are added to the project. In
addition to the software components constituting the compiler and the runtime,
documentations, specifications as well as build, test, setup and evaluation
scripts are important contributions to be covered by this
document.


Regarding the sources, this document is complementing the detailed source code
documentation \cite{insieme_source_doc} generated automatically from the sources
after every successful build. While the source code documentation should serve
the purpose of understanding local implementation details, parameters and
algorithms realized by an easily assessable amount of code (e.g. a single
function), the role of this document is to cover the course grained architecture
of individual components and modules as well as the involved design decisions.
The goal is to provide sufficient inside to enable the reader to

\begin{enumerate}
  \item use the documented components for their intended purpose
  \item customize / extend existing modules for new research ideas
  \item debug components in case problems have been encountered
  \item contribute new modules fitting the project's requirements
\end{enumerate}

While the first point merely requires the reader to understand the end user
facade, the remaining depend on a more detailed insight. Especially the last
objective requires a firm and extensive understanding of the system and the
involved philosophy and standards -- which are tried to be conveyed by this
document. However, although an adequate level of details should be provided
throughout this report, for brevity it is generally assumed that the reader is
familiar with advanced software engineering terminology and concepts as well as
patterns, functional programming, test driven development and C/C++ specific
constructs in particular including memory management, inheritance, templates,
C++ template metaprogramming and boost related topics. Furthermore, for runtime
related issues, a firm background regarding the development and
implementation of parallel codes might turn out to be useful.

\section{How to Use this Document}
This document is intended for for both, basic users striving on utilizing the
Insieme infrastructure to promote their own research as well as developers
improving existing components or contributing new features and capabilities to
the project.

\subsection{As a User}
For a basic user, this documentation is provide an overview over the
offered components and their capabilities. Each chapter starts by a general
overview providing an orientation regarding the available components and
hints covering the necessary steps to combine them to build working solutions.
Therefore, those sections should be studied carefully while the remaining
sections are meant to be used as a reference in case more detailed information
on individual components is required.

However, since the compiler ultimately is just a collection of utilities, hence
a library, glue code connecting the various involved components has to be
implemented by the end-user when aiming on conducting compiler based research.
Within the Insieme project this kind of end-user code is referred to as a
\textit{driver}. To simplify the task of implementing a driver, the interface of
each module shell always be designed in a way such that the required code to
combine existing functionality is as minimal and straight forward as possible.
An example driver implementation including all essential steps is covered in
the corresponding chapter (\ref{cap:compiler:sec:overview:sub:building}).


\subsection{As a Developer}
All users are be encouraged to explore, combine and maybe extend the existing
utilities for their own purposes. If extensions prove to provide a valuable
contribution, those should be feed back into the common repository. Also, new
components extending the capabilities of the Insieme infrastructure are welcome.
This document has been established to supporting both tasks.

For once, the detailed description of the individual components within this
report shell provide enough insight for developers to improve or extend existing
components. Furthermore, all those components shell provide an inside on how to
structure new components. Beside the mere implementation of the additional
features, every component shell be properly isolated from the rest of the
system, documented and tested. Further, especially in the context of the
compiler, the interface offered to users aiming on integrating the new
capability shell be designed intuitively matching the standards established by
the remaining system. Finally, new features shell always be documented within
this report.
