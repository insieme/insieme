\section{The Simple Backend}
There used to be a backend generating code using a thin \textit{pthread} based
runtime layer. This backend, the so called simple backend, which was originally
intended as a prototype for other backends, especially for the runtime backend
generating code for the Insieme Runtime, has been gradually extended and
maintained to support the full spectrum of the IR. However, eventually in August
2012 it has been abendend due to the mature development state of the successors.

The latest version of the simple backend can be found under revision 
\begin{center}
90748c115c8dfb1175dc7703e6d3c5c73f40d287
\end{center}
within the git. The code is organized within its own module (sub-folder
\file{simple\_backend}).

Structually the SimpleBackend is a predecessor of the real backend (see section
\ref{sec:Compiler.Backend}). The core conversion works quite similar. The entire
conversion is build around the \type{StatementConverter} and Utility classes
like the \type{TypeManager}, \type{FunctionManager} or \type{JobManager} are
contributing required services. 

Unlike the real backend, the resulting code is not represented as a C-AST. The
SimpleBackend is directly generating a \type{string} representation of the
resulting C-Code. Although using this approach modifications to the generated
output code are easier to be integrated, post-conversion transformations or the
elimination of unnecessary parentheses (due to the existing precidence of
operators) could not be supported well. Therefore the readability of the
resulting code is limited.


